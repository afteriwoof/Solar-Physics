\input amstex
\documentstyle{amsppt}
\define\el{\dfrac{(x-h)^2}{a^2}+\dfrac{(y-k)^2}{b^2}=1}
\define\bs{$\blacksquare$}
\define\nl{\newline}
\NoBlackBoxes	
\NoRunningHeads
	
\topmatter 
\title Finding Ellipses and Hyperbolas Tangent to two, three, or four given Lines \endtitle 
\author Alan Horwitz \endauthor
\affil Penn State University \endaffil
\address 25 Yearsley Mill Rd., Media, PA 19063 \endaddress
\email alh4\@psu.edu \endemail
\date 5/12/02 \enddate
\keywords ellipse, hyperbola, tangent, quadrilateral \endkeywords
\abstract Given lines $L_{j},\;j=1,2,3,4,$ in the plane, such that no three of the lines are parallel or are concurrent, we want to find the locus of centers of ellipses tangent to the $L_j$. In the case when the lines form the boundary of a four sided convex polygon $R,$ let $M_{1}$ and $M_{2}$ be the midpoints of the diagonals of $R$. Let $L$ be the line thru $M_{1}$ and $M_{2}$, let $Z$ be the open line segment connecting $M_{1}$ and $M_{2},$ let $Y$ be the closed line segment connecting $M_{1}$ and $M_{2},$ and let $X$ be the open line segment which is the part of $L$ lying inside $R.$ It is well known that if an ellipse $E$ is \bf inscribed \rm in $R$, then the center of $E$ must lie on $Z$(see \cite{1} and \cite{2}). We prove(Theorem 11) that every point of $Z$ is the center of some ellipse inscribed in $R,$ which implies that the locus of centers of ellipses inscribed in $R$ is precisely equal to $Z.$ \ In addition, we prove(Theorem 11) that there is a hyperbola tangent to each of the $L_{j}$ and with center $(h,k)\in R$ if and only if $(h,k)\in X-Y$. More generally, any ellipse tangent to the $L_{j}$(and not just inscribed ones) must have its center on $L$. \endabstract \endtopmatter

\document



\bf Introduction \rm \linebreak
	Given finitely lines $L_{j},\;j=1,2,3, ...,N$ in the plane, such that no three of the lines are parallel or are concurrent, we want to find the locus of centers of ellipses tangent to the $L_j$. Our main results concern four given lines (see \S 3), though we need results for two or three given tangents (see \S 1, \S 2, and \S 4). Most of the theorems extend easily to hyperbolas as well. It is useful to make the following definition. \nl \nl
\bf Definition. \rm Given a finite set of distinct lines $L_1,L_2,...,L_N$ in the plane, and an angle $\alpha ,\, -\frac{\pi }{2}<\alpha <\frac{\pi }{2}$, suppose that there is an ellipse with center $(h,k)$ and rotation angle $\alpha $ which is tangent to each of the $L_j$. Then we say that $(h,k)$ is $\alpha $ \bf admissible \rm. If $\alpha =0$ we just call $(h,k)$ admissible. 
	We allow the angle of rotation to \bf vary\rm, and then look at the union, over $\alpha $, of the $\alpha $ admissible centers, $S_{\alpha }$. In the case when the lines form the boundary of a four sided convex polygon $R$, let $M_1$ and $M_2$ be the midpoints of the diagonals of $R$. Let $L$ be the line thru $M_1$ and $M_2$, let $Z$ be the open line segment connecting $M_1$ and $M_2,$ let $X$ be the open line segment which is the part of $L$ lying inside $R,$ and let $Y$ equal the \bf closed \rm line segment connecting $M_1$ and $M_2.$ It is well known that if an ellipse $E$ is \bf inscribed \rm in $R$, then the center of $E$ must lie on $Z$(see \cite{1} and \cite{2}). We prove the stronger result(Theorem 11) that there is an ellipse inscribed in $R$ with center $(h,k)$ if and only if $(h,k)\in Z.$ In addition, we prove(Theorem 11) that there is a hyperbola tangent to each of the $L_{j}$ and with center $(h,k)\in R$ if and only if $(h,k)\in X-Y$. \nl	
\bf Remark \rm: Due to a miscommunication the paper "Finding an Ellipse Tangent to finitely many given Lines"\cite{3}, which appeared in SWJPM in December of 2000, was not the final version to be published. Section 2 is a modification and simplification of results which appeared in \cite{3}. Section 4 contains a simplified and abridged version of material which appeared in \cite{3}. It also contains corrections for numerous errors in the original version. All of our results in Section 3 for four given tangents are new. \nl \nl
\bf \S 1. Preliminary Material \rm \nl
	Our results are stated and proved first for ellipses with major and minor axes parallel to the $x$ and $y$ axes. Later we indicate how to extend the results easily to rotated ellipses. We use the fact that at any point of the ellipse $\el$, $dy/dx=\left( -\dfrac{b^{2}}{a^{2}}\right) \left( \dfrac{x-h}{y-k}\right)$. \nl 
\nl \bf Lemma 1. \rm Let $(h,k)$ be given. Suppose that $L$ is a non-vertical line with equation 
$y-k=m(x-h)+B$\nl
(i) If $(u,v)$ is a \bf positive \rm solution of the equation $m^{2}u+v=B^{2}$, then $B\neq 0,$ and $L$ is tangent to the ellipse $\el$ at $(r,s)$, where $a^{2}=u,b^{2}=v$, $r=h-\dfrac{a^{2}m}{B},$ and $s=k+\dfrac{b^{2}}{B}.$ \nl 
(ii) If $L$ is tangent to the ellipse $E$: $\el$ at $(r,s),$ then $B\neq 0,s\neq k$, and $m^{2}a^{2}+b^{2}=B^{2}.$ \nl
\bf Proof. \rm Since $u,v>0,$ $B\neq 0.$ $\dfrac{(r-h)^{2}}{a^{2}}+\dfrac{(s-k)^{2}}{b^{2}}=\dfrac{a^{2}m^{2}+b^{2}}{B^{2}}=1$ and $m(r-h)+B=m\left( -\dfrac{a^{2}m}{B}\right) +B=\dfrac{-a^{2}m^{2}+B^{2}}{B}=\dfrac{b^{2}}{B}=s-k.$ Thus $(r,s)$ lies on both $E$ and $L$. Note that $s\neq k,$ and since $B\neq 0,\dfrac{r-h}{s-k}=\dfrac{-a^{2}m}{b^{2}}$. Thus $\left[ \dfrac{dy}{dx}\right] _{x=r,y=s}=\dfrac{-b^{2}(r-h)}{a^{2}(s-k)}=m,$ and $L$ is tangent to $E$ at $(r,s)$, which proves (i). If $B=0$, then $L$ passes thru $(h,k),$ which is impossible if $L$ is tangent to an ellipse with center $(h,k)$. If $s=k,$ then $L$ is vertical. Since $dy/dx=\left( -\dfrac{b^{2}}{a^{2}}\right) \left( \dfrac{x-h}{y-k}\right)$ at any point of $E(y\neq k)$, we have 
$$m=-\dfrac{b^{2}}{a^{2}}\frac{r-h}{s-k}\tag{1}$$ Also, $$s-k=m(r-h)+B\tag{2}$$ Using \thetag{1} and substituting  \thetag{2} gives $a^{2}m\left(m(r-h)+B\right) +b^{2}(r-h)=0$, which implies $$r=h-\frac{a^{2}mB}{a^{2}m^{2}+b^{2}}\tag{3}$$ By \thetag{2} we also have $$s=k+\frac{b^{2}B}{a^{2}m^{2}+b^{2}}\tag{4}$$ \thetag{3}, \thetag{4}, and $\dfrac{(r-h)^{2}}{a^{2}}+\dfrac{(s-k)^{2}}{b^{2}}=1$ imply that 
$\dfrac{a^{2}m^{2}B^{2}+b^{2}B^{2}}{(a^{2}m^{2}+b^{2})^{2}}=1$, which implies that 
$B^{2}=a^{2}m^{2}+b^{2}.$ That proves (ii).  $\blacksquare$ \nl \nl
\nl \bf Proposition 1. \rm Suppose that $L_{1}$ and $L_{2}$ are non-vertical lines which are tangent to the ellipse $\el$ at $(r_{j},s_{j}),\;j=1,2$. Suppose that $L_{j}$ has equation 
$$y-k=m_{j}(x-h)+b_{j}\tag{5}$$ $j=1,2$, with $m_{1}^{2}\neq m_{2}^{2}.$ Then 
$$a^{2}=\dfrac{b_{2}^{2}-b_{1}^{2}}{m_{2}^{2}-m_{1}^{2}},\;b^{2}=\dfrac{b_{1}^{2}m_{2}^{2}-b_{2}^{2}m_{1}^{2}}{m_{2}^{2}-m_{1}^{2}}\tag{6}$$
\nl \bf Proof. \rm By Lemma 1, part (ii), $a^{2}=u$ and $b^{2}=v,$ where $u$ and $v$ satisfy the \bf nonsingular \rm linear system 
$$\split
m_{1}^{2}u+v=b_{1}^{2}\\
m_{2}^{2}u+v=b_{2}^{2}\endsplit\tag{7}$$
Substituting back for the unique solution $(u,v)$ yields \thetag{6}. $\blacksquare$ 
\nl \bf Remark. \rm Throughout the paper we suppress, in our notation, the fact that if $y=L_{j}(x)=m_{j}x+c_{j}$ and $y-k=m_{j}(x-h)+b_{j}$ represent the same line, then $$b_{j}=L_{j}(h)-k=m_{j}h+c_{j}-k\tag{8}$$ is really a function of $h$ and $k.$ \nl \nl
\nl \bf Theorem 1. \rm Let $h$ and $k$ be given real numbers, and let $L_{1}$ and $L_{2}$ be distinct, non--vertical lines with equations $y-k=m_{j}(x-h)+b_{j},\;j=1,2,$ $m_{1}^{2}\neq m_{2}^{2}$. \newline 
\bf Part 1: \rm If $$\dfrac{b_{2}^{2}-b_{1}^{2}}{m_{2}^{2}-m_{1}^{2}}>0\text{ and }\dfrac{b_{1}^{2}m_{2}^{2}-b_{2}^{2}m_{1}^{2}}{m_{2}^{2}-m_{1}^{2}}>0\tag{9}$$ then there is a unique ellipse $E$(non-rotated), with center $(h,k),$ which has $L_{1}$ and $L_{2}$ as tangents. Furthermore, $E$ has equation $\dfrac{(x-h)^{2}}{a^{2}}+\dfrac{(y-k)^{2}}{b^{2}}=1,$ where $a^{2}$ and $b^{2}$ are given by \thetag{6}.  \newline 
\bf Part 2:  \rm \thetag{9} is also a \it necessary \rm condition for $L_{1}$ and $L_{2}$ to be tangent to some ellipse(non-rotated), with center $(h,k)$. 
\nl \bf Proof. \rm Part 1: Since \thetag{9}  holds, \thetag{7} has the unique positive solution $(u,v)$ with $u=a^{2}$ and $v=b^{2}$, $a^{2}$ and $b^{2}$ defined as in \thetag{6}.  By Lemma 1, part (i), the $L_{j}$ are tangent, at $(r_{j},s_{j})$, to the ellipse $E$ with equation $\el$, where $r_{j}=h-\dfrac{a^{2}m_{j}}{b_{j}}$ and $s_{j}=k+\dfrac{b^{2}}{b_{j}}.$ Now if $\tilde{E}$ is \it any \rm ellipse with center $(h,k)$ and tangent to the $L_{j}$, then the equation of $\tilde{E}$ is $\dfrac{(x-h)^{2}}{\tilde{a}^{2}}+\dfrac{(y-k)^{2}}{\tilde{b}^{2}}=1$. By Proposition 1, $\tilde{a}$  and $\tilde{b}$ are given by \thetag{6}, which implies that $\tilde{a}=a^{2}$ and $\tilde{b}^{2}=b^{2}.$ That proves uniqueness. \nl
Part 2: Follows immediately from Proposition 1. $\blacksquare$ \newline Given $L_{1}$ and $L_{2},$ we can now give an equivalent characterization of the set of admissible centers. \nl 
\nl \bf Corollary 1. \rm Let $y=L_{1}(x)$ and $y=L_{2}(x)$ be distinct, non--vertical lines with equations $y=L_{j}(x)=k+m_{j}(x-h)+b_{j},j=1,2$ and $\left| m_{1}\right| <$ $\left| m_{2}\right| $. Then there is an ellipse $E$(non-rotated) with center $(h,k),$ which has $L_{1}$ and $L_{2}$ as tangents if and only if 
$$\left| \dfrac{m_{1}}{m_{2}}\right| \left| b_{2}\right| <\left| b_{1}\right| <\left| b_{2}\right|\tag{10} $$\newline or equivalently $$\left| \dfrac{m_{1}}{m_{2}}\right| <\left| \dfrac{L_{1}(h)-k}{L_{2}(h)-k}\right| <1\tag{11}$$ 
\nl \bf Proof. \rm Since $\left| m_{1}\right| <$ $\left| m_{2}\right| ,$ \thetag{10} is equivalent to \thetag{9}. Since $b_{j}=$ $L_{j}(h)-k,$ \thetag{10} is equivalent to \thetag{11} and the fact that $b_{2}\neq 0\Leftrightarrow L_{2}(h)-k\neq 0.$ $\blacksquare$ 
\nl \bf Example. \rm $L_{1}:\;y=x+3$ and $L_{2}:\;y=2x-3$ are given. Then $L_{1}$ and $L_{2}$ are tangents to a non-rotated ellipse $E$ with center $(h,k)$ if and only if $\dfrac{1}{2}<\left| \dfrac{h+3-k}{2h-3-k}\right| <1.$
\nl \bf Remark. \rm One can also obtain results similar to Proposition 1 or Theorem 1 if one of the $L_{j}$(but not \bf both \rm) is \it vertical \rm. If $L_{1}$ has equation $x-h=b_{1},$ then $b_{1}=a$ or $-a$, which implies that $a^{2}=$ $b_{1}^{2}$. Since $m_{2}^{2}a^{2}+b^{2}=b_{2}^{2}$ still holds, one then solves to get $b^{2}$. \nl We shall have occasion to use the following theorem, which follows immediately from Lemma 1, Proposition  1, and Theorem 1.
\nl \nl \bf Theorem 2. \rm  Given $N$ non-vertical lines $L_{1},...,T_{N}$ with equations $y-k=m_{j}(x-h)+b_{j},\;j=1,...,N.$ \newline (i) There is an ellipse $E$(non-rotated)$,$ with center $(h,k),$ tangent to each of the $L_{j}$ if and only if the linear system $$\split m_{1}^{2}u+v=b_{1}^{2}\\ m_{2}^{2}u+v=b_{2}^{2}\\ \vdots\\m_{N}^{2}u+v=b_{N}^{2}\endsplit\tag{12}$$ has a \bf positive \rm solution $(u,v).$ If $m_{i}^{2}\neq m_{j}^{2}$ whenever $i\neq j$, then $E$ is unique and has equation $\dfrac{(x-h)^{2}}{a^{2}}+\dfrac{(y-k)^{2}}{b^{2}}=1,$ where $a^{2}=u$ and $b^{2}=v.$ \newline (ii) If $m_{i}^{2}\neq m_{j}^{2}$ whenever $i\neq j$ and \thetag{12} is consistent, then for any $i\neq j,$ the unique solution of \thetag{12} is $$u=\dfrac{b_{j}^{2}-b_{i}^{2}}{m_{j}^{2}-m_{i}^{2}}\text{ and }v=\dfrac{b_{i}^{2}m_{j}^{2}-b_{j}^{2}m_{i}^{2}}{m_{j}^{2}-m_{i}^{2}}\tag{13}$$
\nl \bf Remark. \rm If $m_{1}^{2}=m_{2}^{2}$, then the existence-uniqueness result above fails. We discuss this for two given lines in the next theorem.
\nl \nl \bf Theorem 3. \rm  Let $h$ and $k$ be given real numbers, and let $L_{1}$ and $L_{2}$ be distinct, non--vertical lines with equations $y-k=m_{j}(x-h)+b_{j},\;j=1,2.$ Suppose that $m_{1}^{2}=m_{2}^{2}$. \newline (i) If $\left| b_{1}\right| =\left| b_{2}\right| \neq 0,$ then $L_{1}$ and $L_{2}$ are tangent to any ellipse $E\in \digamma =\left\{ \dfrac{(x-h)^{2}}{a^{2}}+\dfrac{(y-k)^{2}}{b^{2}}=1\right\} \;,$ where $a^{2}$ and $b^{2}$ are any positive real numbers satisfying the equation $$a^{2}m_{1}^{2}+b^{2}=b_{1}^{2}\tag{14}$$
(ii) If $\left| b_{1}\right| \neq \left| b_{2}\right| $ or $b_{1}=0=b_{2},$ then $L_{1}$ and $L_{2}$ are not tangent to any ellipse(non-rotated) with center $(h,k)$.
\nl \bf Proof. \rm Since the line with equation $m_{1}^{2}u+v=b_{1}^{2}$ has a positive $v$ intercept and a non-positive slope, $m_{1}^{2}u+v=b_{1}^{2}$ has infinitely many positive solutions in the unknowns $u$ and $v$. If $a^{2}=u,b^{2}=v$ for some positive solution $(u,v)$, then \thetag{14}  holds. Also, $a^{2}m_{2}^{2}+b^{2}=b_{2}^{2}$ since $m_{1}^{2}=m_{2}^{2}$ and $b_{1}^{2}=b_{2}^{2}.$ (i) then follows from Lemma 1, part (i). To prove (ii), suppose that $L_{1}$ and $L_{2}$ are tangent to an ellipse $E$ with equation $\dfrac{(x-h)^{2}}{a^{2}}+\dfrac{(y-k)^{2}}{b^{2}}=1$. By Lemma 1, part (ii), $b_{j}^{2}=a^{2}m_{j}^{2}+b^{2},\;j=1,2$. Lemma 1, part (ii), $m_{1}^{2}=m_{2}^{2},$, and $b_{1}^{2}=b_{2}^{2},$ then imply that $b_{1}^{2}\neq 0\neq b_{2}^{2}$. That contradicts the assumptions in (ii). $\blacksquare$ \newline 
\nl \bf Rotated Ellipses \rm \nl
	For simplicity of exposition, we have only considered non-rotated ellipses. However, the previous results extend with little effort to ellipses whose axes are \it rotated \rm clockwise about $(0,0)$ thru a specified angle $\alpha ,-\dfrac{\pi }{2}<\alpha <\dfrac{\pi }{2}$. Let $(x_{\alpha },y_{\alpha })$ denote the new coordinates, and suppose that we are given $(h,k)$ in $xy$ coordinates. Assume throughout that no given line is parallel to the $y_\alpha$ axis(i.e., all lines are non--vertical in the rotated coordinates). It follows easily that the line $y-k=m(x-h)+b$ becomes $$y_{\alpha }-k_{\alpha }=\left( \frac{m+\tan \alpha }{1-m\tan \alpha }\right) (x_{\alpha }-h_{\alpha })+\frac{b\sec \alpha }{1-m\tan \alpha }\tag{15}$$ where $h_{\alpha }=h\cos \alpha -k\sin \alpha ,k_{\alpha }=h\sin \alpha +k\cos \alpha $ and $1-m\tan \alpha \neq 0$. We find it useful to introduce the following notation. For given $m_{i,\alpha },b_{i,\alpha }, 
m_{j,\alpha },b_{j,\alpha },$ and $\alpha $: $$S_{ij}=\dfrac{b_{j,\alpha }^{2}-b_{i,\alpha }^{2}}{m_{j,\alpha }^{2}-m_{i,\alpha }^{2}},T_{ij}=\dfrac{b_{i,\alpha }^{2}m_{j,\alpha }^{2}-b_{j,\alpha }^{2}m_{i,\alpha }^{2}}{m_{j,\alpha }^{2}-m_{i,\alpha }^{2}}\tag{16}$$ We now state the following generalization of Theorem 1 without proof.
\nl \nl \bf Theorem 4. \rm  Let $h_\alpha$ and $k_\alpha$ be given real numbers, and let $L_{1}$ and $L_{2}$ be distinct lines with equations $y_{\alpha }-k_{\alpha }=m_{j,\alpha }(x_{\alpha }-h_{\alpha })+b_{j,\alpha },\;j=1,2$.  Suppose that $m_{2,\alpha }^{2}\neq m_{1,\alpha }^{2}$. \newline Part 1: If $$S_{12}>0\text{ and }T_{12}>0\tag{17}$$ then there is a unique ellipse $E$, with equation $\dfrac{(x_{\alpha }-h_{\alpha })^{2}}{a^{2}}+\dfrac{(y_{\alpha }-k_{\alpha })^{2}}{b^{2}}=1,$ which has $L_{1}$ and $L_{2}$ as tangents, where $a^{2}=S_{12}$ and $b^{2}=T_{12}$ \newline Part 2: \thetag{17} is also a \it necessary \rm condition for $L_{1}$ and $L_{2}$ to be tangent to some ellipse with center $(h,k)$ and rotation angle $\alpha $. \nl 
\bf Remark. \rm As in the non-rotated case, if $\left| m_{1,\alpha }\right| <$ $\left| m_{2,\alpha }\right| ,$ then  \thetag{17} is equivalent to $$\left| \dfrac{m_{1,\alpha }}{m_{2,\alpha }}\right| \left| b_{2,\alpha }\right| <\left| b_{1,\alpha }\right| <\left| b_{2,\alpha }\right| \tag{18}$$ It is also interesting to allow the rotation angle to \it vary \rm. This leads to the following \newline \bf Question: \rm Let $L_{1}$ and $L_{2}$ be two given nonparallel lines. 
Is any point $(h,k)\notin L_{1}\cup L_{2}\;\alpha $ admissible for some $\alpha $, $-\dfrac{\pi }{2}<\alpha <\dfrac{\pi }{2}$ ? The following theorem answers this question in the affirmative. \nl \nl
\bf Theorem 5. \rm  Let $h$ and $k$ be given real numbers, and let $L_{1}$ and $L_{2}$ be distinct, non--vertical lines with equations $y-k=m_{j}(x-h)+b_{j},\;j=1,2.$ Assume that $m_{1}\neq m_{2}$ and that $(h,k)\notin L_{1}\cup L_{2}$. Then there is an angle $\alpha ,-\dfrac{\pi }{2}<\alpha <\dfrac{\pi }{2},$ such that the ellipse $E$ with center $(h,k)$ and rotation angle $\alpha $ has $L_{1}$ and $L_{2}$ as tangents. \nl 
\bf Proof. \rm Note that $(h,k)\notin L_{1}\cup L_{2}\Rightarrow b_{1}\neq 0\neq b_{2}.$ We want to apply \thetag{10} of Corollary 1 in the \it new \rm coordinates $x_{\alpha }$ and $y_{\alpha }$. Using \thetag{15} , we replace $m_{j}$ by $m_{j,\alpha }$ and $b_{j}$ by $b_{j,\alpha }$, where  $m_{j,\alpha }=\dfrac{m_{j}+\tan \alpha }{1-m_{j}\tan \alpha }$ and $b_{j,\alpha }=\dfrac{b_{j}\sec \alpha }{1-m_{j}\tan \alpha }$ Since we are allowing the angle of rotation to vary, we may rotate the coordinate axes so that in the new coordinates, $m_{1}=0<m_{2}.$ Then $\left| \dfrac{m_{1}}{m_{2}}\right| \left| b_{2}\right| <\left| b_{1}\right| .$ If $\left| b_{1}\right| <\left| b_{2}\right| ,$ then by the corollary to Theorem 1, there is a non-rotated ellipse tangent to the $L_{j}.$ So assume now that $\left| b_{2}\right| \leq \left| b_{1}\right| $, and let $\alpha _{0}=\arctan (1/m_{2})$. Note that $\dfrac{m+\tan \alpha }{1-m\tan \alpha }$ is a strictly increasing function of $m,$ which implies that $m_{1,\alpha }<m_{2,\alpha }$ for any given $\alpha $. As $\alpha \rightarrow \alpha _{0},\dfrac{\left| b_{1,\alpha }\right| }{\left| m_{1,\alpha }\right| }=$ $\left| \dfrac{b_{1}\sec \alpha }{\tan \alpha }\right| \rightarrow $ $\left| \dfrac{b_{1}\sec \alpha _{0}}{\tan \alpha _{0}}\right| =$ $\left| b_{1}\right| \sqrt{1+m_{2}^{2}},$ and $\dfrac{\left| b_{2,\alpha }\right| }{\left| m_{2,\alpha }\right| }=$ $\left| \dfrac{b_{2}\sec \alpha }{m_{2}+\tan \alpha }\right| \rightarrow $ $\left| \dfrac{b_{2}\sec \alpha _{0}}{m_{2}+\tan \alpha _{0}}\right| =$ $\dfrac{\left| b_{2}\right| }{\sqrt{1+m_{2}^{2}}}.$ Note that $\tan \alpha _{0}=\dfrac{1}{m_{2}}\Rightarrow m_{2}+\tan \alpha _{0}\neq 0.$ Now $\dfrac{\left| b_{2,\alpha }\right| }{\left| m_{2,\alpha }\right| }<\dfrac{\left| b_{1,\alpha }\right| }{\left| m_{1,\alpha }\right| }$ if and only if $\dfrac{\left| b_{2}\right| }{\sqrt{1+m_{2}^{2}}}<\left| b_{1}\right| \sqrt{1+m_{2}^{2}}$ if and only if $\left| \dfrac{b_{2}}{b_{1}}\right| <1+m_{2}^{2},$ which holds since $\left| \dfrac{b_{2}}{b_{1}}\right| \leq 1<1+m_{2}^{2}.$ Also, $\lim\limits_{\alpha \rightarrow \alpha _{0}}\left| b_{2,\alpha }\right| =\lim\limits_{\alpha \rightarrow \alpha _{0}}\left| \dfrac{b_{2}\sec \alpha }{1-m_{2}\tan \alpha }\right| =\infty ,$ while $\lim\limits_{\alpha \rightarrow \alpha _{0}}\left| b_{1,\alpha }\right| =\lim\limits_{\alpha \rightarrow \alpha _{0}}\left| b_{1}\sec \alpha \right| \neq \infty .$ Thus for $\alpha $ sufficiently close to $\alpha _{0}$, $\left| \dfrac{m_{1,\alpha }}{m_{2,\alpha }}\right| \left| b_{2,\alpha }\right| <\left| b_{1,\alpha }\right| <\left| b_{2,\alpha }\right| ,$ and the theorem follows from the remark after Theorem 4.  $\blacksquare$ 
\nl \nl \bf An Aside on Hyperbolas \rm  \newline 
	For \bf hyperbolas \rm rather than ellipses tangent to $L_{1}$ and $L_{2},$ one would want a solution of \thetag{7} with $uv<0.$ In Lemma 1, part (ii), if $T$ is tangent to the hyperbola $H:\dfrac{(x-h)^{2}}{a^{2}}-\dfrac{(y-k)^{2}}{b^{2}}=\pm 1$ at $(r,s),$ then $B\neq 0,s\neq k$, and $m^{2}a^{2}-b^{2}=\pm B^{2}.$ It then follows that a necessary and sufficient condition to have a hyperbola tangent to $L_{1}$ and $L_{2}$ is $\dfrac{b_{2}^{2}-b_{1}^{2}}{m_{2}^{2}-m_{1}^{2}}\dfrac{b_{1}^{2}m_{2}^{2}-b_{2}^{2}m_{1}^{2}}{m_{2}^{2}-m_{1}^{2}}<0.$ A version of Theorem 4 for hyperbolas follows immediately. \nl \nl 
\bf Theorem 6. \rm  Let $h_\alpha$ and $k_\alpha$ be given real numbers, and let $L_{1}$ and $L_{2}$ be distinct lines with equations $y_{\alpha}-k_{\alpha}=m_{j,\alpha}(x_{\alpha}-h_{\alpha})+b_{j,\alpha},\;j=1,2.$ Assume that $m_{2,\alpha}^{2}\neq m_{1,\alpha }^{2}$. \newline Part 1: If $$S_{12}T_{12}<0\tag{19}$$ then there is a unique hyperbola $H$, with center $(h,k)$ and rotation angle $\alpha,$ which has $L_{1}$ and $L_{2}$ as tangents. \newline Part 2: \thetag{19} is also a \it necessary \rm condition for a hyperbola with center $(h,k)$ and rotation angle $\alpha $ to exist which has $L_{1}$ and $L_{2}$ as tangents. \newline \nl
\bf \S 2 Rotated Versions of Key Theorems from \S 1 \rm \newline 
	Assume throughout, unless stated otherwise, that no given line is parallel to the $y_\alpha$ axis. The line $L_{j}$ with equation $y=m_{j}x+c_{j}$ becomes $$y_{\alpha}=m_{j,\alpha}x_{\alpha}+c_{j,\alpha}\tag{20}$$ where $$m_{j,\alpha}=\dfrac{m_{j}+\tan \alpha}{1-m_{j}\tan \alpha}\text{ and }c_{j,\alpha}=\dfrac{c_{j}\sec \alpha}{1-m_{j}\tan \alpha }\tag{21}$$ $1-m_{j}\tan \alpha \neq 0$. Rewrite $L_{j}$ in the form $$y_{\alpha}-k_{\alpha}=m_{j,\alpha}(x_{\alpha}-h_{\alpha})+b_{j,\alpha}, b_{j,\alpha}=m_{j,\alpha}h_{\alpha}+c_{j,\alpha}-k_{\alpha}\tag{22}$$ In $xy$ coordinates, $h_{\alpha}=h\cos \alpha -k\sin \alpha$ and $k_{\alpha}=h\sin \alpha +k\cos \alpha$ for some real numbers $h$ and $k$. Dividing thru by $\cos \alpha $ and using the substitution $$w=\tan \alpha $$ yields $$h_{\alpha}=\dfrac{1}{\sqrt{1+w^{2}}}(h-kw),k_{\alpha}=\dfrac{1}{\sqrt{1+w^{2}}}(hw+k)\tag{23}$$ which gives $$b_{j,\alpha}=\dfrac{m_{j,\alpha}}{\sqrt{1+w^{2}}}(h-kw)+c_{j,\alpha}-\dfrac{1}{\sqrt{1+w^{2}}}(hw+k)\tag{24}$$ We shall always speak of $\alpha $ admissible centers in $xy$ coordinates. Throughout, given $L_{1},...,L_{N},$ $\Re _{\alpha}$ denotes the set of $\alpha $ admissible centers, i.e., the set of all $(h,k)$ such that the ellipse,$\dfrac{(x_{\alpha}-h_{\alpha})^{2}}{a^{2}}+\dfrac{(y_{\alpha}-k_{\alpha})^{2}}{b^{2}}=1,$ is tangent to each of the $L_{j}$. If $L_{i}$ and $L_{j}$ are not parallel, we let $i(L_{i},L_{j})=(x_{l,\alpha},y_{l,\alpha}),i\neq l\neq j$ denote their point of intersection(in rotated coordinates). Theorems 14-17(proved in \S 4) and Theorem 2  are easily extendable to rotated ellipses, which we state here. \nl \nl
\bf Theorem 7. \rm  Suppose one is given $N$ lines $L_{1},...,L_{N}$ with equation given by \thetag{22} and an angle $\alpha ,-\dfrac{\pi }{2}<\alpha <\dfrac{\pi }{2}.$ Then \newline (i) There is an ellipse $E$ with center $(h,k)$ and rotation angle $\alpha $ tangent to each of the $L_{j}$ if and only if the linear system $$\split m_{1,\alpha}^{2}u+v=b_{1,\alpha}^{2}\\ m_{2,\alpha}^{2}u+v=b_{2,\alpha}^{2}\\ \vdots\\ m_{N,\alpha}^{2}u+v=b_{N,\alpha}^{2}\endsplit\tag{25}$$ has a positive solution $(u,v).$ If $m_{i,\alpha }^{2}\neq m_{j,\alpha }^{2}$ whenever $i\neq j$, then $E$ is unique and the equation of $E$ in the new rotated coordinates is $\dfrac{(x_{\alpha }-h_{\alpha })^{2}}{a^{2}}+\dfrac{(y_{\alpha }-k_{\alpha })^{2}}{b^{2}}=1,$ where $a^{2}=u$ and $b^{2}=v.$ (ii) If $m_{i,\alpha }^{2}\neq m_{j,\alpha }^{2}$ whenever $i\neq j$ and \thetag{25} is consistent, then for any $i\neq j,$ the unique solution of \thetag{25} is (see \thetag{16}) $$u=S_{ij}\text{ and }v=T_{ij}\tag{26}$$ Expanding the determinants and the left-hand side of \thetag{36} in Theorem 12 yields \nl \nl
\bf Theorem 8. \rm  Let $L_{j},\;j=1,2,3,\;$be distinct, non-concurrent lines with equations given by \thetag{20},  $i\neq j\Rightarrow m_{i,\alpha }^{2}\neq m_{j,\alpha }^{2}$. Let $d_{\alpha }=\prod\limits_{j>i}(m_{j,\alpha }-m_{i,\alpha }),$ \newline 
$a_{1,\alpha }=m_{1,\alpha }c_{1,\alpha }\left( m_{3,\alpha }^{2}-m_{2,\alpha }^{2}\right) -m_{2,\alpha }c_{2,\alpha }\left( m_{3,\alpha }^{2}-m_{1,\alpha }^{2}\right) +m_{3,\alpha }c_{3,\alpha }\left( m_{2,\alpha }^{2}-m_{1,\alpha }^{2}\right) $, \newline
$a_{2,\alpha }=-c_{1,\alpha }\left( m_{3,\alpha }^{2}-m_{2,\alpha }^{2}\right) +c_{2,\alpha }\left( m_{3,\alpha }^{2}-m_{1,\alpha }^{2}\right) -c_{3,\alpha }\left( m_{2,\alpha }^{2}-m_{1,\alpha }^{2}\right) $, \newline
$a_{3,\alpha }=(c_{1,\alpha }^{2}(m_{3,\alpha }^{2}-m_{2,\alpha }^{2})-c_{2,\alpha }^{2}(m_{3,\alpha }^{2}-m_{1,\alpha }^{2})+c_{3,\alpha }^{2}(m_{2,\alpha }^{2}-m_{1,\alpha }^{2}))/2.$ \newline Then \thetag{25}, 
with $N=3$, has a unique solution if and only if $(h_{\alpha },k_{\alpha })$ lies on the curve $\gamma $ with equation $$d_{\alpha }h_{\alpha }k_{\alpha }+a_{1,\alpha }h_{\alpha }+a_{2,\alpha }k_{\alpha }+a_{3,\alpha }=0\tag{27}$$ \nl \nl
\bf Theorem 9. \rm  Let $L_{1}$ and $L_{2}\;$be distinct lines with equations given by \thetag{20}, $0\neq m_{1,\alpha }^{2}\neq m_{2,\alpha }^{2}$. Let $L_{3}$ be the line with equation $x_{\alpha }=c_{3,\alpha }.$ Let $\gamma $ be the curve with equation \newline $(h_{\alpha }-x_{3,\alpha })(k_{\alpha }-a_{1,\alpha })=-\frac{1}{2}\left( m_{1,\alpha }+m_{2,\alpha }\right) (c_{3,\alpha }-x_{3,\alpha })^{2},$ where  \newline $a_{1,\alpha }=L_{2}(c_{3,\alpha })+L_{1}(c_{3,\alpha })-y_{3,\alpha }.$ Let $q_{3}(k_{\alpha })=\prod\limits_{j=1}^{3}(k_{\alpha }-y_{j,\alpha })$, \newline $q_{1}(k_{\alpha })=m_{1,\alpha }^{2}(k_{\alpha }-a_{1,\alpha }).$ Then $(h,k)\in \Re _{\alpha }\Leftrightarrow (h_{\alpha },k_{\alpha })\in \gamma ,\;h_{\alpha }\neq c_{3,\alpha }\text{ and }\dfrac{q_{3}(k_{\alpha })}{q_{1}(k_{\alpha })}>0$. For each $(h,k)\in \Re _{\alpha }$, there is a unique ellipse $E$ with equation $\dfrac{(x_{\alpha }-h_{\alpha })^{2}}{a^{2}}+\dfrac{(y_{\alpha }-k_{\alpha })^{2}}{b^{2}}=1$ which is tangent to each of the $L_{j}$, with $a^{2}=(h_{\alpha }-c_{3,\alpha })^{2}$ and $b^{2}=\dfrac{q_{3}(k_{\alpha })}{q_{1}(k_{\alpha })}$. \newline \nl
\bf \S 3 Main Results for Four Given Tangents \rm \newline	
	We now discuss the locus of centers of ellipses tangent to \bf four \rm given lines, where the angle of rotation is allowed to vary. In particular, using the methods of this paper we are able to strengthen and extend some known results about ellipses inscribed in quadrilaterals. Given four lines in the plane, $L_{j},\;j=1,2,3,4,$ such that no three of the lines are parallel or are concurrent, we want to find the locus of centers of ellipses tangent to the lines. In the case when the lines form the boundary of a four sided convex polygon $R,$ let $M_{1}$ and $M_{2}$ be the midpoints of the diagonals of $R$. Let $L$ be the line thru $M_{1}$ and $M_{2}$, let $Z$ be the open line segment connecting $M_{1}$ and $M_{2},$ let $Y$ be the \bf closed \rm line segment connecting $M_{1}$ and $M_{2},$ and let $X$ be the open line segment which is the part of $L$ lying inside $R.$ It is well known that if an ellipse $E$ is \bf inscribed \rm in $R$, then the center of $E$ must lie on $Z$(see \cite{1} and \cite{2}). The proof of this result actually goes back to Newton. It is stated in \cite{2} that the locus of centers of ellipses inscribed in $R$ actually \bf equals \rm $Z$, but Newton only proved that the center of $E$ must lie on $Z$, as is noted in \cite{1}. Indeed, it is not even clear that an ellipse even \bf exists \rm which is inscribed in $R,$ let alone whether \bf every point \rm of $Z$ is the center of such an ellipse. We prove(Theorem 11) that every point of $Z$ is the center of some ellipse inscribed in $R,$ which implies that the locus of centers of ellipses inscribed in $R$ is precisely equal to $Z.$ \ In addition, we prove (Theorem 11) that there is a hyperbola tangent to each of the $L_{j}$ and with center $(h,k)\in R$ if and only if $(h,k)\in X-Y$. More generally, any ellipse tangent to the $L_{j}$(and not just inscribed ones) must have its center on $L$. \nl \nl 
\bf Theorem 11. \rm  
Let $R$ be a four sided \bf convex \rm polygon in the $xy$ plane which is not a parallelogram, and let $M_{1}$ and $M_{2}$ be the midpoints of the diagonals of $R$. Let $L$ be the line thru $M_{1}$ and $M_{2}$, and let $L_{j},\;j=1,2,3,4$ denote the lines which make up the boundary of $R$. Let $Z$ be the open line segment connecting $M_{1}$ and $M_{2},$ let $X$ be the open line segment which is the part of $L$ lying inside $R,$ and let $Y$ equal the \bf closed \rm line segment connecting $M_{1}$ and $M_{2}.$ \newline
(i) If an ellipse $E$ or hyperbola $H$ is tangent to each of the $L_{j}$, then the center of $E$ or of $H$ must lie on $L$. \newline
(ii) There is an ellipse $E,$ \bf inscribed \rm in $R$ and with center $(h,k),$ if and only if $(h,k)\in Z.$ \newline
(iii) There is a hyperbola tangent to each of the $L_{j}$ and with center $(h,k)\in R,$ if and only if $(h,k)\in X-Y$.
\nl \bf Remark. \rm If $R$ is a parallelogram, then an ellipse $E$ which is inscribed in $R$ must have center equal to the center of $R$. There is no hyperbola tangent to each of the $L_{j}$ and with center $(h,k)\in R$. 
\nl \bf Proof. \rm We shall assume that no two sides of $R$ are parallel. The proof when at least two sides of $R$ are parallel is similar, but somewhat simpler, than the proof given here. By using an affine transformation, we may assume that the vertices of $R$ are $(0,0),(1,0),$ $(0,1)$, and $(s,t)$ for some real numbers $s$ and $t$. Since $R$ is convex, it follows easily that  $$s>0,t>0\text{ and }s+t\geq 1$$ Since $R$ is four sided and no two sides of $R$ are parallel, $$s+t>1\text{ and }s\neq 1\neq t$$ Write $L_{1}:y=0,L_{2}:x=0,L_{3}:y=1+\left( \dfrac{t-1}{s}\right) x,L_{4}:y=\dfrac{t}{s-1}(x-1)${}. For fixed $\alpha,$ \thetag{21} gives $m_{1,\alpha }=\allowbreak w$, $m_{3,\alpha }=\dfrac{t-1+ws}{s-(t-1)w}$, $m_{4,\alpha }=\allowbreak \dfrac{t+w(s-1)}{s-1-wt}$, $c_{1,\alpha }=\allowbreak 0$, $c_{3,\alpha }=\allowbreak \dfrac{s\sqrt{1+w^{2}}}{s-(t-1)w}$, $c_{4,\alpha }=\allowbreak -t\dfrac{\sqrt{1+w^{2}}}{s-1-wt}.$ The equation of $L_{2}$ in $x_{\alpha }$ and $y_{\alpha }$ coordinates is given by \thetag{20}, with $m_{2,\alpha }=-\dfrac{1}{w}$ and $c_{2,\alpha }=0$. Writing $L_{j}$ in the form $y_{\alpha }-k_{\alpha }=m_{j,\alpha }(x_{\alpha }-h_{\alpha })+b_{j,\alpha }$ yields, by \thetag{24}, $b_{1,\alpha }=\allowbreak -k\sqrt{1+w^{2}},$ \newline $b_{2,\alpha }=\allowbreak -h\dfrac{\sqrt{1+w^{2}}}{w},$ $b_{3,\alpha }=\dfrac{\left( (t-1)h-(k-1)s\right) w^{2}+(t-1)h-(k-1)s}{\left( s-(w-1)t\right) \sqrt{1+w^{2}}},$ \newline $b_{4,\alpha }=\allowbreak \dfrac{\allowbreak \left( t(h-1)-k(s-1)\right) w^{2}+t(h-1)-k(s-1)}{\left( s-1-wt\right) \sqrt{1+w^{2}}}.$ \ In general, \newline $m_{j,\alpha }^{2}-m_{i,\alpha }^{2}=c_{i,j}\allowbreak \left( 1+w^{2}\right) \dfrac{p_{i,j}(w)}{q_{i,j}(w)}$, where the $c_{i,j}$ are nonzero constants depending on $s$ and $t$, and $p_{i,j}$ and $q_{i,j}$ are polynomials depending on $s$ and $t,$ with $\deg $ $p_{i,j}=2$ and $\deg $ $q_{i,j}\leq 4.$ In particular, $m_{2,\alpha }^{2}-m_{1,\alpha }^{2}=\allowbreak \dfrac{1-w^{4}}{w^{2}}$. Now $m_{j,\alpha }$ and $c_{j,\alpha }$ are not defined, for $j\geq 2,$ when $$w\in F_{1}=\{0,\dfrac{s}{t-1},\dfrac{s-1}{t}\}$$ Let $$F_{2}=\{w:m_{j,\alpha }^{2}-m_{i,\alpha }^{2}=0\text{ for some }i\neq j\}$$ and let $$F=F_{1}\cup F_{2},G=\{w:-\infty <w<\infty \}-F$$ We must assume, for now, that $w\in G,$ and in particular we want $m_{j,\alpha }^{2}-m_{i,\alpha }^{2}\neq 0$ so that we can apply Theorem 8. For $s$ and $t$ given, $F$ is a \it finite \rm set. By Theorem 8, applied to $L_{1},L_{2},L_{4}$ and $L_{2},L_{3},L_{4},$ respectively, if $w\in G$, then an admissible center $(h,k)$ must lie on the curves with equations $$d_{\alpha }h_{\alpha }k_{\alpha }+a_{1,\alpha }h_{\alpha }+a_{2,\alpha }k_{\alpha }+a_{3,\alpha }=0\tag{28}$$ and $$D_{\alpha }h_{\alpha }k_{\alpha }+A_{1,\alpha }h_{\alpha }+A_{2,\alpha }k_{\alpha }+A_{3,\alpha }=0\tag{29}$$ where $m_{4,\alpha }$ replaces $m_{3,\alpha }$ when Theorem 8 is applied to $L_{1},L_{2},$ and $L_{4},$ and $m_{4,\alpha }$ replaces $m_{1,\alpha }$ when Theorem 8 is applied to $L_{2},L_{3},$ and $L_{4}.$ In \thetag{29} we used capital letters to better distinguish the two equations. Using \thetag{23}, we can rewrite \thetag{28} and \thetag{29} in the original $xy$ coordinate system to obtain $$d_{\alpha }\allowbreak \tfrac{1}{1+w^{2}}(h-kw)(hw+k)+a_{1,\alpha }\tfrac{1}{\sqrt{1+w^{2}}}(h-kw)+a_{2,\alpha }\tfrac{1}{\sqrt{1+w^{2}}}(hw+k)+\allowbreak a_{3,\alpha }=0\tag{30}$$ and $$D_{\alpha }\allowbreak \tfrac{1}{1+w^{2}}(h-kw)(hw+k)+A_{1,\alpha }\tfrac{1}{\sqrt{1+w^{2}}}(h-kw)+A_{2,\alpha }\tfrac{1}{\sqrt{1+w^{2}}}(hw+k)+\allowbreak A_{3,\alpha }=0\tag{31}$$ Note that $d_{\alpha }$ and $D_{\alpha }$ have the factor $m_{4,\alpha }-m_{2,\alpha }$ in common. Thus multiplying \thetag{30} by $\left( m_{4,\alpha }-m_{3,\alpha }\right) \left( m_{3,\alpha }-m_{2,\alpha }\right) $, multiplying \thetag{31} by $\left( m_{4,\alpha }-m_{1,\alpha }\right) \left( m_{2,\alpha }-m_{1,\alpha }\right) ,$ and subtracting yields 
$$\left( 1+w^{2}\right) ^{4}st(t-1+s)(w^{2}(s-1)+2wt-w^{2}+1) \frac{2k(s-1)-2h(t-1)-(s-t)}{2(s-w(t-1))^{2}w^{3}(s-1-wt) ^{3}}=0$$ which implies that $2k(s-1)-2h(t-1)-(s-t)=0,$ or $$k=L(h)\allowbreak =\frac{1}{2}\dfrac{s-t+2h(t-1)}{s-1}\tag{32}$$ Now $M_{1}=\left( \dfrac{1}{2},\dfrac{1}{2}\right) $ and $M_{2}=\left( \dfrac{1}{2}s,\dfrac{1}{2}t\right) $, so that $L(x)$ from \thetag{32} is the line $L$ thru $M_{1}$ and $M_{2}.$ Note that the \bf same equations \rm must be satisfied in order to have a \bf hyperbola \rm  tangent to all four lines. Also, while we have assumed that $w\in G,$ \thetag{32} actually holds for any $w\in (-\infty ,\infty ).$ One can prove this directly for $w\in F_{1}$ using Theorem 9. It also follows easily that for any given $w$, there is \it at most one pair \rm $i\neq j$ with $m_{j,\alpha }^{2}-m_{i,\alpha }^{2}=0.$ One can then choose two sets of three of the $L_{j}$ so that $L_{i}$ and $L_{j}$ never appear in the same set. \thetag{32}  can then be derived in the same way as above. That proves (i). We now attempt to solve \thetag{30}  with $k=L(h).$ First, the LHS of \thetag{30} factors into $\dfrac{1}{4}\dfrac{\left( 1+w^{2}\right) ^{2}t}{w^{2}\left( s-1-tw\right) ^{2}(s-1)}P(w,h)$, where $$P(w,h)=A(h)w^{2}+B(h)w-A(h)$$ $A(h)=2\left( s-1\right) \left( \left( 2t-2\right) h^{2}+\left( s+2-t\right) h-s\right) $ and \newline $B(h)=\left( 1-2h\right) \left( s-t\right) \left( \left( 2s-4+2t\right) h+s-t\right) .$ Of course the coefficients of \newline $P(w,h)$ depend on $s$ and $t$ as well, but we suppress that in our notation. \nl 	
\bf Claim: \rm If $k=L(h)$, then $A(h)$ and $B(h)$ cannot both be $0$ for the same value of $h.$ \nl 	
\bf Proof of Claim: \rm Suppose that $A(h_{0})=B(h_{0})=0$ for some $h_{0}$. Now $B(h_{0})=0$ if and only if $h_{0}=\dfrac{1}{2}$ or $h_{0}=\dfrac{1}{2}\dfrac{t-s}{s+t-2}$. If $h_{0}=\dfrac{1}{2}$, then $\left( 2t-2\right) h_{0}^{2}+\left( s+2-t\right) h_{0}-s=\allowbreak \dfrac{1}{2}-\dfrac{1}{2}s\neq 0$ since $s\neq 1$. If $h_{0}=\dfrac{1}{2}\dfrac{t-s}{s+t-2}$, then $k_{0}=L(h_{0})=\allowbreak \dfrac{1}{2}\dfrac{s-t}{s-2+t}=-h_{0}$, which implies that $h_{0}$ and $k_{0}$ cannot both be positive. That proves the claim. \nl By the claim, $P$ has two \bf real \rm roots $w=w_{0}$ and $w=-\dfrac{1}{w_{0}}$ for each given $h,$ whenever $w_{0}\neq 0$, i.e., whenever $\left( 2t-2\right) h^{2}+\left( s+2-t\right) h-s\neq 0$(note that $s-1\neq 0$). Of course this just reflects the fact that one can rotate the coordinates axes by $\alpha $ or by $\alpha -\dfrac{\pi }{2}$ and obtain the same axes of the ellipse. The LHS of \thetag{31}, with $k=L(h),$ factors into $-\dfrac{1}{4}\dfrac{s\left( 1+w^{2}\right) ^{2}\left( s-1+t\right) }{w^{2}\left( s-wt+w\right) ^{2}\left( s-1-wt\right) ^{2}\left( s-1\right) }P(w,h).$ Thus \thetag{30}  and \thetag{31} hold simultaneously for the same $h,k,$ and $w$ if and only if $P(w,h)=0$ and $k=L(h)$. Since $w\in G,$ $\ m_{j,\alpha }^{2}-m_{i,\alpha }^{2}\neq 0$ for $i\neq j.$ By Theorem 8, \thetag{28}  and \thetag{29} hold simultaneously for the same $h_{\alpha },k_{\alpha },$ and $w$ if and only if \thetag{25} has a solution, with $N=4$. Then, by Theorem 7, part (ii), \thetag{25} has a solution, with $N=4,$ if and only if $$S_{ij}=S_{lm},\text{ }T_{ij}=T_{lm}\text{ for any distinct }i,j,l,m\in \{1,2,3,4\}\tag{33}$$ Of course, \thetag{28}  and \thetag{29} are equivalent to \thetag{30}  and \thetag{31} in $xy$ coordinates. Now an ellipse exists tangent to all four lines if and only if the same ellipse can be found which is tangent to $L_{1},L_{2},$ and $L_{4},$ and to $L_{2},L_{3},$ and $L_{4}$. Thus Theorem 7 and Theorem 8 imply that an ellipse exists tangent to all four lines if and only if there are $w\geq 0$ and $h$ such that $P(w,h)=0$,$\;S_{ij}>0,$ and $T_{ij}>0$ for some $i\neq j$, with $k=L(h).$ It is easiest to work with $i=1, j=2,$ which is possible except for $w=0$ or $1$. We want to prove that if $(h,k)\in Z$, with $P(w,h)=0,$ then $S_{12}$ and $T_{12}$ are \bf both positive \rm if $w\notin \{0,1\}.$ We shall prove that at least one of $S_{12}$ and $T_{12}$ are positive at $M_{1}$ and $M_{2}$, and that neither can be zero on $Z.$ Let $$I=\text{open interval with endpoints }\dfrac{1}{2}\text{ and }\dfrac{1}{2}s$$ For each given $h\in I,$ we shall always assume that $w_{0}$ denotes the \bf nonnegative \rm root of $P(w,h).$ Then $w_{0}$ is a continuous function of $h$, which we denote by $f(h)$. Letting $w=f(h)$ and $k=L(h),$ $S_{ij}$ and $T_{ij}$ are then functions of $h$ on $I$. $S_{12}(h)$ and $T_{12}(h)$ are continuous at any point of $I$ if $w=f(h)\neq 0$ or $1$. We now show how to extend $S_{12}$ and $T_{12}$ to be continuous at those points as well. First, $(h,k)\in Z$ if and only if $h=\lambda \dfrac{1}{2}+(1-\lambda )\dfrac{1}{2}s,0<$ $\lambda <1$ and $k=L(h)$. We now break the rest of the proof up into two cases: \newline \bf Case 1: \rm $s\neq t$ \newline $w=1$ is a solution of $P(w,h)=0$ if and only if $B(h)=0,$ which holds if and only if $h=\dfrac{1}{2}$ or $h=\dfrac{1}{2}\dfrac{t-s}{s+t-2}.$ If $h=\dfrac{1}{2}$, then $(h,k)\notin Z$ and $(h,k)\notin X-Y$(we shall use the latter fact when proving (iii)). If $h=\dfrac{1}{2}\dfrac{t-s}{s+t-2}$, then $k=L(h)=\allowbreak \dfrac{1}{2}\dfrac{s-t}{s-2+t}=-h$, which implies that $h$ and $k$ cannot both be positive. Hence $w=1$ \bf cannot \rm occur as a solution of $P(w,h)=0$ for $(h,k)\in Z\cup X-Y$. However, $P(0,h_{0})$ can equal $0$ for $h_{0}\in I$ and certain values of $s$ and $t$. This is not a problem since, for general $h$ and $k,S_{12}=\dfrac{k^{2}w^{2}-h^{2}}{w^{2}-1}$ and $T_{12}=\dfrac{h^{2}w^{2}-k^{2}}{w^{2}-1},w\notin \{0,1\}.$ Thus for \it any \rm fixed $h$ and $k$, $\lim\limits_{w\rightarrow 0}S_{12}=h^{2}$ and $\lim\limits_{w\rightarrow 0}T_{12}=\allowbreak k^{2}$. $S_{12}$ and $T_{12}$ are then continuous at $h_{0}$ if we define, with $k=L(h),$ $S_{12}(h_{0})=h_{0}^{2}$ and $T_{12}(h)=(L(h_{0}))^{2}$. It is also possible that as $h\rightarrow h_{0}\in I,$ the positive root of $P(w,h)$ approaches $\infty .$ Since $\lim\limits_{w\rightarrow \infty }$ $S_{12}=k^{2}$ and $\lim\limits_{w\rightarrow \infty }$ $T_{12}=h^{2},$ defining, with $k=L(h),$ $S_{12}(h_{0})=(L(h_{0}))^{2}$ and $T_{12}(h_{0})=h_{0}^{2}$ makes $S_{12}$ and $T_{12}$ continuous at $h_{0}$ in that case. \newline \bf Case 2: \rm $s=t$ \newline In this case the line $L$ is $y=x,$ which contains one of the diagonals. The proof is very similar, but simpler, than the proof above for $s\neq t$. Either $\allowbreak \left( 2t-2\right) h^{2}+2h-t=0,$ which gives an inscribed circle, or $w=\pm 1$ and $h=k$ can take on any value between $\dfrac{1}{2}$ and $\dfrac{1}{2}t$. \nl Thus we have proven that $S_{12}$ and $T_{12}$ are continuous at any point of $I$. Now consider the endpoints of $I$. \newline 	
$\bullet $ If $h=\dfrac{1}{2}s$, then $P(w,h)=-\left( s-1\right) ^{2}\left( ws+t\right) \left( s-tw\right) =0\Leftrightarrow w=\dfrac{s}{t}\neq 1$ since $s\neq t$. If $w=\dfrac{s}{t}$, then $S_{12}=\allowbreak 0$ and $T_{12}=\dfrac{1}{4}t^{2}+\dfrac{1}{4}s^{2},$ and thus $S_{12}$ and $T_{12}$ are continuous and nonnegative at $\dfrac{1}{2}s$. \nl 	
$\bullet $ If $h=\dfrac{1}{2}$, then $P(w,h)=-2(w^2-1)\left( s-1\right) ^{2}=0\Leftrightarrow w=1$ since $s\neq 1.$ $S_{12}$ and $T_{12}$ are undefined when $w=1$. However, if $w=1$ and $h=\dfrac{1}{2}$, then $S_{13}=\allowbreak \dfrac{1}{2}$ and $T_{13}=\allowbreak 0$. By \thetag{33}, this implies that $\lim\limits_{h\rightarrow 1/2}$ $S_{12}(h)=\dfrac{1}{2}$ and $\lim\limits_{h\rightarrow 1/2}T_{12}(h)=0.$ \linebreak Thus defining $S_{12}\left( \dfrac{1}{2}\right) =\dfrac{1}{2}$ and $T_{12}\left( \dfrac{1}{2}\right) =0$ extends $S_{12}$ and $T_{12}$ to be continuous and nonnegative at $\dfrac{1}{2}$. \nl Thus $S_{12}$ and $T_{12}$ are continuous(or can be extended to be continuous) on the closure of $I,$ and nonnegative at the endpoints of $I$. 
\newline \bf Claim: \rm $S_{12}(h)$ and $T_{12}(h)$ cannot equal $0$ for $h\in I$. 
\newline \bf Proof of Claim: \rm We can assume that $w\neq 1$ since we have shown that $w=1$ \bf cannot \rm occur as a solution of $P(w,h)=0$ for $h\in I.$ If $P(0,h)=0$, then $S_{12}(h)$ and $T_{12}(h)$ cannot equal $0$ since $h^{2}\neq 0$, $L^{2}(h)\neq 0$ on $I.$ So assume also that $w\neq 0$. Then $S_{12}=\dfrac{k^{2}w^{2}-h^{2}}{w^{2}-1}=\allowbreak 0$ if and only if $h=kw$($h\neq -kw$ since $h>0,k>0$). If $h=kw$ along with $k=L(h)$, then $h=\dfrac{1}{2}w\dfrac{s-t}{s-1-(t-1)w}$ if $w\neq \dfrac{s-1}{t-1}$, which implies that $P(w,h)=-2(w^2-1) \left( s-1\right) ^{2}\dfrac{s-tw}{s-1-(t-1)w}$, which equals $0$ if and only if $w=\dfrac{s}{t}$, which implies that $h=\dfrac{1}{2}s\notin I$. If $h=kw\;$along with $k=L(h)$ and $w=\dfrac{s-1}{t-1}$, then $h=\dfrac{1}{2}\dfrac{s-t+2h(t-1)}{t-1}$, which has no solution since $s-t\neq 0$. $T_{12}=0$ if and only if $k=wh$. If $k=wh$ along with $k=L(h)$, then $h=\allowbreak $ $\dfrac{1}{2}\dfrac{s-t}{w(s-1)-(t-1)}$ if $w\neq \dfrac{t-1}{s-1}$. This implies that $P(w,h)=-2\left( w+1\right) \left( w-1\right) ^{2}\left( s-1\right) ^{2}\dfrac{(s^{2}-s)w+t-t^{2}}{\left( w(s-1)-(t-1)\right) ^{2}}=0$ if and only if $w=\dfrac{t(t-1)}{s(s-1)}$, which implies that $h=\allowbreak -\dfrac{1}{2}\dfrac{s}{t-1}$. Now $h>0\Rightarrow t<1$ and $w\geq 0\Rightarrow s<1$. Also, $(h,k)\in Z$ if and only if $h=\lambda \dfrac{1}{2}+(1-\lambda )\dfrac{1}{2}s,0<$ $\lambda <1$ and $k=L(h)$. Thus $-\dfrac{1}{2}\dfrac{s}{t-1}=\lambda \dfrac{1}{2}+(1-\lambda )\dfrac{1}{2}s\Rightarrow \lambda =\dfrac{s\allowbreak t}{\left( t-1\right) \left( s-1\right) }.$ Since $s+t\geq 1$, $t\geq 1-s$ and $1-t\leq s\Rightarrow \lambda =\dfrac{s\allowbreak }{1-s}\dfrac{\allowbreak t}{1-t}\geq \dfrac{s\allowbreak }{1-s}\dfrac{1-s}{s}=1,$ which implies that $h\notin I$. If $k=wh$, along with $k=L(h)$ and $w=\dfrac{t-1}{s-1}$, then again there is no solution since $s-t\neq 0$. That proves the claim. 
\newline Since $S_{12}(h)$ and $T_{12}(h)$ are nonnegative at the endpoints of $I,$ positive at at least one of the endpoints of $I,$ and nonzero on the interior of $I$, by the Intermediate Value Theorem $S_{12}(h)>0$ and $T_{12}(h)>0$ when $h\in I.$ Now let $H=\{h\in I:f(h)\in G\}$. $(-\infty ,\infty )-G$ is a finite set and $P(w,h)=A(h)(w^{2}+C(h)w-1),$where $C(h)$ is a nontrivial rational function. Thus $P(w_{0},h_{0})=0$ if and only if $C(h_{0})=\dfrac{1}{w_{0}}-w_{0},$ and by the definition of $f$, $w_{0}=f(h_{0}).$ It then follows easily that $f^{-1}\left\{ (-\infty ,\infty )-G\right\} $ is also a finite set, which implies that $I-H$ is a finite set. We have proven that if $h\in H$ and $k=L(h),$ then there is an ellipse with equation $\dfrac{(x_{\alpha }-h_{\alpha })^{2}}{a^{2}}+\dfrac{(y_{\alpha }-k_{\alpha })^{2}}{b^{2}}=1$ which is tangent to each of the $L_{j},j=1,2,3,4$, where $a^{2}=S_{12}(h)$ and $b^{2}=T_{12}(h)$. Since $S_{12}(h)$ and $T_{12}(h)$ are both positive on $int(I), $if $h_{0}\in I-H,$ one can obtain an ellipse with center $(h_{0},L(h_{0}))$ by taking a limit of the ellipses obtained for $h\in H$. That proves that if $(h,k)\in Z$, then there is an ellipse inscribed in $R$ and with center $(h,k).$ 
\newline \bf Rest of the proof of (ii) and (iii) \rm \newline Note that $(h,k)\in X\Rightarrow h,k>0$ and that $\{h:(h,k)\in X-Y\}$ consists of two disjoint open intervals, $J_{1}\cup J_{2}$. The particular endpoints of $X$ depend on whether $s>1,s<1,t>s,$ etc. 
\newline \bf Claim: \rm $S_{12}(h)$ and $T_{12}(h)$ cannot equal $0$ for $h\in J_{1}\cup J_{2}$.
\newline \bf Proof of Claim: \rm The proof follows exactly as in the proof above that $S_{12}(h)$ and $T_{12}(h)$ cannot equal $0$ for $h\in I,$ except for showing that $T_{12}(h)\neq 0$ when $w\neq \dfrac{t-1}{s-1},$ $h=\allowbreak \dfrac{1}{2}\dfrac{s}{1-t},$ and $s,t<1$. We omit the proof when $s=t$ and consider the following two cases. \newline \bf Case 1: \rm $s<t$ \newline Then $L(x)$ intersects $\partial R$ in the lines $L_{2}$ and $L_{4},J_{1}=\left( 0,\dfrac{1}{2}s\right) ,$ and \newline $J_{2}=\left( \dfrac{1}{2},\dfrac{1}{2}s+\dfrac{1}{2}t\right) $. Now $\dfrac{1}{2}s+\dfrac{1}{2}t<\dfrac{1}{2}\dfrac{s}{1-t}$ since $s+t>1,$ which implies that $h=\allowbreak \dfrac{1}{2}\dfrac{s}{1-t}\notin J_{1}\cup J_{2}$ since $\dfrac{1}{2}s+\dfrac{1}{2}t$ is the largest value of $h\in J_{1}\cup J_{2}$. \newline \bf Case 2: \rm $s>t$ \newline Then $L(x)$ intersects $\partial R$ in the lines $L_{1}$ and $L_{3},J_{1}=\left( \dfrac{1}{2}\dfrac{t-s}{t-1},\dfrac{1}{2}s\right) ,$ and \newline $J_{2}=\left( \dfrac{1}{2},\dfrac{1}{2}s\dfrac{s-2+t}{t-1}\right) .$ Again, $s+t>1$ implies that $\dfrac{1}{2}s\dfrac{s-2+t}{t-1}<\dfrac{1}{2}\dfrac{s}{1-t},$ which implies that $h=\allowbreak \dfrac{1}{2}\dfrac{s}{1-t}\notin J_{1}\cup J_{2}$ since $\dfrac{1}{2}s\dfrac{s-2+t}{t-1}$ is the largest value of $h\in J_{1}\cup J_{2}$. That proves the claim. \newline We want to show that $S_{12}(h)$ and $T_{12}(h)$ change sign at $h=\dfrac{1}{2}s$ and at $h=\dfrac{1}{2},$ respectively. Recall that $w=f(h)\Leftrightarrow $ $P(w,h)=A(h)w^{2}+B(h)w-A(h)=0$ and $w\geq 0$, where $A(h)=2\left( s-1\right) \left( \left( 2t-2\right) h^{2}+\left( s+2-t\right) h-s\right) $ and $B(h)=\left( 1-2h\right) \left( s-t\right) \left( \left( 2s-4+2t\right) h+s-t\right) $. Thus implicit differentiation gives $w^{\prime }=\dfrac{dw}{dh}=-\dfrac{A^{\prime }(h)w^{2}+B^{\prime }(h)w-A^{\prime }(h)}{2wA(h)+B(h)}.$ Assume throughout that $k=L(h).$ Now suppose that $S_{12}^{\prime }(h)=0.$ We shall derive a contradiction. Since $S_{12}=\dfrac{k^{2}w^{2}-h^{2}}{w^{2}-1}$, $$(h^{2}-k^{2}) ww^{\prime }+(kk^{\prime }w^{2}-h) (w^{2}-1)=0\tag{34}$$ where $k^{\prime }=\dfrac{dk}{dh}=L^{\prime }(h)=\dfrac{t-1}{s-1}.$ First, if $h=\dfrac{1}{2}s$, then $w=\dfrac{s}{t}\Rightarrow \dfrac{dw}{dh}=\allowbreak 2\left( s-t\right) \dfrac{s^{2}-2s+t^{2}-2t}{t^{2}\left( s-1\right) \left( s^{2}+t^{2}\right) }$, and \thetag{34} becomes $w( \allowbreak 2( s-t) \frac{s^{2}-2s+t^{2}-2t}{t^{2}( s-1) ( s^{2}+t^{2}) }) (h^{2}-k^{2})+(w^{2}-1)( k( \frac{t-1}{s-1}) w^{2}-h) =$ $-\left( s+t\right) ^{2}\left( s-t\right) ^{2}\dfrac{s}{t^{3}\left( s^{2}+t^{2}\right) \left( s-1\right) }=0,$ which it cannot since $s\neq t$. Thus $S_{12}^{\prime }( \frac{1}{2}s) \neq 0$, which implies that $S_{12}(h)$ changes sign at $h=\dfrac{1}{2}s.$ Second, if $h=\dfrac{1}{2}$, then $w=1,$ and $T_{12}(h)$ is undefined when $w=1$. However, consider $T_{12}^{\prime }(h)=\dfrac{(w^{2}-1)(w^{2}h-kk^{\prime })+ww^{\prime }k^{2}}{(w^{2}-1)^{2}}.$ Now $w^{\prime }(\frac{1}{2}) =-\dfrac{B^{\prime }(\frac{1}{2})}{2A( \frac{1}{2}) +B( \frac{1}{2}) }=\allowbreak -2\dfrac{s-t}{s-1},$ which implies that $\lim\limits_{h\rightarrow 1/2,w\rightarrow 1}(w^{2}-1)(w^{2}h-kk^{\prime })+ww^{\prime }k^{2}=\dfrac{1}{4}w^{\prime }( \frac{1}{2}) =\dfrac{1}{2}\dfrac{t-s}{s-1}\neq 0$. Thus $T_{12}^{\prime }(h)$ has the same sign on either side of $h=\dfrac{1}{2}$, which implies that $T_{12}(h)$ also changes sign at $h=\dfrac{1}{2}.$ Since $S_{12}( \frac{1}{2}s) =T_{12}( \frac{1}{2}) =0,$ $S_{12}( \frac{1}{2}) >0,$, $T_{12}( \frac{1}{2}s) >0,$ and $S_{12}(h),T_{12}(h)>0$ on $I$, by the claim above and the Intermediate Value Theorem, $S_{12}(h)T_{12}(h)<0$ on $J_{1}\cup J_{2}$. Let $J=\{h\in J_{1}\cup J_{2}:f(h)\in G\}$. We have shown that there is a hyperbola with center $(h,k)$ tangent to each of the $L_{j}$ if $h\in J$. Again one can use a limiting argument, as earlier, if $h\notin J.$ That proves that if $(h,k)\in X-Y,$ then there is a hyperbola with center $(h,k)$ tangent to each of the $L_{j}$. To finish the proof of (ii) and (iii): In the proof of (ii) given above, look at the \bf non-positive \rm roots $w$ of $P(w,h)$ instead. Everything follows in the same way. Thus, if $h\in I,$ then $S_{ij}(h)>0$ and $T_{ij}(h)>0$ for either choice of the root $w.$ Thus there cannot be a hyperbola with center $(h,k)\in Z$ which is tangent to each of the $L_{j}$. Similarly, in the proof just given that if $(h,k)\in X-Y,$ then there is a hyperbola with center $(h,k)$ tangent to each of the $L_{j}$, again look at the non-positive roots $w$ of $P(w,h).$ If $h\in J_{1}\cup J_{2},$ then $S_{ij}(h)T_{ij}(h)<0,$ which implies that if $(h,k)\in X-Y$, then there cannot be an ellipse inscribed in $R$ and with center $(h,k).$ \nl   
\bf Remark. \rm The fact that if $(h,k)\in X-Y$, then there cannot be an ellipse inscribed in $R$ and with center $(h,k)$ follows from the result in \cite{1} mentioned earlier. However, we wanted to give a self contained proof which does not use orthogonal projections. \newline It is possible that one is just given the four lines $L_{j},\;j=1,2,3,4$ and not the polygon $R$ of which they form the boundary. One still might want to know if there is some ellipse tangent to the given lines, and to characterize the locus of centers of ellipses tangent to the given lines. The following proposition gives a partial answer. \nl \nl
\bf Proposition 2. \rm Let $L_{1},L_{2},L_{3},L_{4}$ be four given lines in the $xy$ plane, such that no three of the $L_{j}$ are concurrent. Suppose also that one of the following holds: \newline 
(i) No \bf two \rm of the $L_{j}$ are parallel, or \nl	
(ii) Exactly two of the $L_{j}$ are parallel, and the intersection point of the other two lines does not lie between the two parallel lines. \nl Then the $L_{j}$ form the boundary of a four sided convex polygon $R$. \nl
\bf Proof. \rm (i) Pick a line $L$. The other lines intersect $L$ at three distinct points since no three lines have a common intersection point. Let $M$ be the line which intersects $L$ between the other two intersection points. The interior of the triangle $T$ formed by the three lines not equal to $M$ is cut by $M$. The two regions into which $M$ divides $T$ are each convex, since they are the intersections of convex regions(a triangle and a half plane). One of these regions is $R$. \nl
(ii) Suppose that $L_{1}$ and $L_{2}$ are parallel and let $P_{ij}=$ point of intersection of $L_{i}$ and $L_{j},(i,j)\neq (1,2)$. Let $R$ be the polygon with vertices $P_{13},P_{14},$ $P_{23},$ and $P_{24}$. Since $P_{34}$ does not lie between $L_{1}$ and $L_{2},$ it follows easily that $R$ is a four sided convex polygon. $\blacksquare$ \nl 	Theorem 11 then gives \nl
\bf Corollary 2. \rm If $L_{1},L_{2},L_{3},L_{4}$ satisfy (i) or (ii) of Proposition 2, then there is an ellipse, $E$, tangent to each of the $L_{j},$ and the center of $E$ must lie on the line thru the midpoints of the diagonals of $R.$ \nl \nl
\bf \S 4. Three Given Tangents \rm \newline
	The rest of the paper is a modification and simplification of results which appeared in \cite{3}. It also contains corrections for several errors in the original version. \nl \nl Throughout, given $L_{1},...,L_{N},$ $\Re $ denotes the set of admissible centers, i.e., the set of all $(h,k)$ such that the \bf non-rotated \rm ellipse,$\dfrac{(x-h)^{2}}{a^{2}}+\dfrac{(y-k)^{2}}{b^{2}}=1,$ is tangent to each of the $L_{j}$. For convenience, we use the following notation from \thetag{16} for $\alpha =0.$ For given $m_{i},b_{i},m_{j},$ and $b_{j}$: $$S_{ij}=\dfrac{b_{j}^{2}-b_{i}^{2}}{m_{j}^{2}-m_{i}^{2}},T_{ij}=\dfrac{b_{i}^{2}m_{j}^{2}-b_{j}^{2}m_{i}^{2}}{m_{j}^{2}-m_{i}^{2}}$$ We want to characterize $\Re $ when $N=3$ and also give an explicit formula for $E.$ Several cases must be considered depending upon whether $m_{i}^{2}=m_{j}^{2}$ and/or $m_{i}=m_{j}$ for some $i\neq j$. In each case the admissible centers lie on a hyperbola or a straight line(a degenerate hyperbola). The precise curve is obtained by finding conditions on the coefficient matrix of the linear system \thetag{12} with $N=3$. If $L_{i}$ and $L_{j}$ are not parallel, we let $$(x_{l},y_{l})=\left( \frac{c_{j}-c_{i}}{m_{i}-m_{j}},\frac{m_{i}c_{j}-m_{j}c_{i}}{m_{i}-m_{j}}\right) \tag{35}$$ $i\neq l\neq j,$ denote their point of intersection. \nl
\nl \bf \S 4.1 No two of the tangents have slopes equal in absolute value \rm \newline
We first state our results when none of the $L_{j}$ is horizontal or vertical. 
\nl \nl \bf Theorem 12. \rm Let $L_{1},L_{2},L_{3}\;$be distinct, non-concurrent, and non--vertical lines with equations $y=L_{j}(x)=m_{j}x+c_{j}$, $j=1,2,3$. Assume that $i\neq j\Rightarrow m_{i}^{2}\neq m_{j}^{2}$ and let $D=\prod\limits_{j>i}(m_{j}-m_{i})$, $M=\left( \matrix m_1 & 1 & c_1\\ m_2 & 1 & c_2\\ m_3 & 1 & c_3   \endmatrix \right)$,  
$a_{1}=\Bigg \vert \matrix 1 & m_1^2 & m_1c_1\\1 & m_2^2 & m_2c_2\\1 & m_3^2 & m_3c_3\endmatrix\Bigg \vert $,  
$a_{2}=-\Bigg \vert \matrix 1 & m_1^2 & m_1\\1 & m_2^2 & m_2\\1 & m_3^2 & m_3\endmatrix\Bigg \vert $,  
$p_{3}(h)=\prod\limits_{j=1}^{3}(h-x_{j})$,$\;p_{1}(h)=h+\dfrac{a_{2}}{D}$, $q_{3}(k)=\prod\limits_{j=1}^{3}(k-y_{j})$, $q_{1}(k)=k+\dfrac{a_{1}}{D}$, $C=-\dfrac{\vert M \vert ^2}{2D^2}\prod\limits_{j>i}(m_{j}+m_{i})$. Let $\gamma $ be the curve with equation $$p_{1}(h)q_{1}(k)=C\tag{36}$$ (i) Then $$\Re =\left\{ (h,k)\in \gamma :\dfrac{p_{3}(h)}{p_{1}(h)}>0\text{ and }\dfrac{q_{3}(k)}{q_{1}(k)}>0\right\}\tag{37} $$ For each $(h,k)\in \Re $, there is a unique ellipse $E:\dfrac{(x-h)^{2}}{a^{2}}+\dfrac{(y-k)^{2}}{b^{2}}=1$ which is tangent to each of the $L_{j}$, with $a^{2}=$ $\dfrac{p_{3}(h)}{p_{1}(h)}$and $b^{2}=$ $\dfrac{q_{3}(k)}{q_{1}(k)}.$ \nl
(ii) Write equation \thetag{36} in the form $k=f(h)$ or $h=g(k)$. \nl
(a) Suppose that $m_{3}\neq 0,$ let $w_{j}=g(y_{j})$,$\;j=1,2,3,$ and define the cubic polynomial $$r_{3}(h)=-2\dfrac{x_{2}-x_{3}}{m_{3}-m_{2}}\prod\limits_{j=1}^{3}m_{j}\prod\limits_{j=1}^{3}(h-w_{j})$$ Then $$\Re =\left\{ (h,f(h)):\frac{p_{3}(h)}{p_{1}(h)}>0\text{ and }r_{3}(h)>0\right\} \tag{38}$$ with $a^{2}$ as in (i) and $b^{2}=\dfrac{r_{3}(h)}{(p_{1}(h))^{2}}.$ \nl
(b) Suppose that $m_{3}=0$. Then $$\Re =\left\{ (h,k)\in \gamma :\frac{p_{3}(h)}{p_{1}(h)}>0\text{ and }h\neq \dfrac{1}{2}(x_{1}+x_{2})\right\} \tag{39}$$ with $a^{2}$ as in (i) and $b^{2}=(f(h)-c_{3})^{2}$ \nl
(iii) Also, given any $0\leq e_{0}<1$, there is an $(h,k)\in \Re $ such that $E$ has eccentricity $e_{0}$. \nl
\bf Proof. \rm Note that rank$(M)\geq 2.$ If $\vert M\vert=0$, then 
rank$(M)=2=$ rank$\!\left( \matrix m_1 & 1\\ m_2 & 1\\m_3 & 1\endmatrix \right)$, which would imply that the system of equations $m_{j}x+c_{j}=y,j=1,2,3$ has a solution $(x,y).$ Since we assumed that the $L_{j}$ are \bf not \rm concurrent, $\vert M \vert\neq 0$. This implies that $C\neq 0$. Since $\vert M \vert =(m_{l}-m_{i})(m_{j}-m_{i})(x_{j}-x_{l})$ for any distinct $\{i,j,l\}\subset \{1,2,3\}$, $$x_{i}\neq x_{j}\text{ for }i\neq j\tag{40}$$ It is also useful to note that $$\vert M \vert =-c_{2}m_{3}+c_{2}m_{1}+c_{1}m_{3}+c_{3}m_{2}-c_{3}m_{1}-c_{1}m_{2}\neq 0\tag{41}$$ Given $(h,k)$, write $L_{j}$ in the form $y-k=m_{j}(x-h)+b_{j}$, $j=1,2,3$. By Theorem 2 with $N=3$, there is an ellipse, with center $(h,k)$, tangent to \bf all three \rm lines if and only if $$\split m_{1}^{2}u+v=b_{1}^{2}\\ m_{2}^{2}u+v=b_{2}^{2}\\ m_{3}^{2}u+v=b_{3}^{2}\endsplit\tag{42}$$ has a positive solution $(u,v)$. Since rank$\!\left( \matrix m_1^2 & 1\\ m_2^2 & 1\\m_3^2 & 1\endmatrix \right)=2$, rank$\!\left(\matrix m_1^2 & 1 & b_1^2\\ m_2^2 & 1 & b_2^2\\m_3^2 & 1 & b_3^2\endmatrix \right)\geq 2,$ and thus \thetag{42} has a solution(not necessarily positive) if and only if $\Bigg\vert\matrix m_1^2 & 1 & b_1^2\\ m_2^2 & 1 & b_2^2\\m_3^2 & 1 & b_3^2\endmatrix\Bigg\vert =0,$ which gives $$(m_{3}^{2}-m_{2}^{2})b_{1}^{2}-(m_{3}^{2}-m_{1}^{2})b_{2}^{2}+(m_{2}^{2}-m_{1}^{2})b_{3}^{2}=0\tag{43}$$ By Theorem 2 again, if \thetag{43}  holds, then the solution of \thetag{42} is unique and is given by $$u=S_{12}=S_{23}=S_{13}\tag{44}$$ and $$v=T_{12}=T_{23}=T_{13}\tag{45}$$ Hence, by Theorem 2, with $N=3,$ $$S_{12}>0\text{ and }T_{12}>0\tag{46}$$ and \thetag{43}  are necessary and sufficient conditions for the existence of an ellipse $E$ tangent to all three lines. Using \thetag{8}, \thetag{43} becomes $\Bigg\vert\matrix m_{1}^{2} & (m_{1}h+c_{1}-k)^{2} & 1\\m_{2}^{2} & (m_{2}h+c_{2}-k)^{2} & 1\\m_{3}^{2} &  (m_{3}h+c_{3}-k)^{2} & 1\endmatrix\Bigg\vert =0.$ Multiplying column 1 by $-h^{2}$ and adding to column 2 yields $\Bigg\vert\matrix m_{1}^{2} & 2m_{1}h(c_{1}-k)+(c_{1}-k)^{2} & 1\\m_{2}^{2} & 2m_{2}h(c_{2}-k)+(c_{2}-k)^{2} & 1\\m_{3}^{2} & 2m_{3}h(c_{3}-k)+(c_{3}-k)^{2} & 1\endmatrix\Bigg\vert =0$, which is equivalent to \thetag{36}. Solving  \thetag{36}  for $k$ and for $h$, respectively, yields $$k=f(h)=\dfrac{C}{p_{1}(h)}-\dfrac{a_{1}}{D}\tag{47}$$ and $$h=g(k)=\dfrac{C}{q_{1}(k)}-\dfrac{a_{2}}{D}\tag{48}$$ It is easy to show that$,$ for any distinct $\{i,j,l\}\subset \{1,2,3\}$, $$Dx_{j}+a_{2}=(m_{i}+m_{l})\vert M \vert\tag{49}$$ and $$Dy_{j}+a_{1}=-m_{j}(m_{i}+m_{l})\vert M \vert\tag{50}$$ By \thetag{40},\thetag{49}, and \thetag{50}, $Dx_{l}+a_{2}\neq 0$ and $Dy_{l}+a_{1}\neq 0$ for any $l$. Hence by \thetag{47} and \thetag{48}, $v_{j}=f(x_{j})$ and $w_{j}=g(y_{j})$ are each \it finite \rm. By \thetag{8}, $S_{12}=p(h,k),$ where $p$ is a polynomial which is monic and quadratic in $h$, and linear in $k$. Also by \thetag{8}, $$b_{j}^{2}-b_{i}^{2}=\left( L_{j}(h)-T_{i}(h)\right) \left( L_{j}(h)+T_{i}(h)-2k\right)\tag{51}$$ Thus $h=x_{3}\Rightarrow L_{2}(h)=L_{1}(h)\Rightarrow b_{2}^{2}-b_{1}^{2}=0\Rightarrow p(x_{3},k)=0$ for any $k.$ Now write $L_{j}$ in the form $x-h=\dfrac{1}{m_{j}}(y-k)-\dfrac{b_{j}}{m_{j}}$. Substituting $\dfrac{1}{m_{j}}$ for $m_{j}$ and $-\dfrac{b_{j}}{m_{j}}$ for $b_{j}$ in $S_{12}$ yields $T_{12}.$ By interchanging $h$ and $k,$ this easily implies that $T_{12}=q(h,k),$ where $q$ is a polynomial which is monic and quadratic in $k$, and linear in $h$. Also, $q(h,y_{3})=0$ for any $h$. Finally, by \thetag{44} and \thetag{45}, $p(x_{j},k)=q(h,y_{j})=0$ for any $h$ and $k$, and for $j=1,2,3.$ For $(h,k)\in \gamma,$ $p(h,k)=p(h,f(h))=\dfrac{p_{3}(h)}{p_{1}(h)}$, where 
$p_{3}$ is a monic polynomial of degree $\leq$ $3$ and $p_{1}(h)=h+\dfrac{a_{2}}{D},$ by \thetag{47}. 
$p(x_{j},f(x_{j}))=0$ since $p(x_{j},k)=0$ for any $k$, which implies that $p_{3}(h)=\prod\limits_{j=1}^{3}(h-x_{j}).$ Similarly, for $(h,k)\in \gamma ,$ $q(h,k)=q(g(k),k)=\dfrac{q_{3}(k)}{q_{1}(k)}$, where $q_{3}$ is a monic polynomial of degree $\leq$ $3$ and $q_{1}(k)=k+\dfrac{a_{1}}{D},$ by \thetag{48}. $q(g(y_{j}),y_{j})=0$ since $q(h,y_{j})=0$ for any $h$, which implies that $q_{3}(k)=\prod\limits_{j=1}^{3}(k-y_{j})$. Since $S_{12}=\dfrac{p_{3}(h)}{p_{1}(h)}$ and $T_{12}=\dfrac{q_{3}(k)}{q_{1}(k)}$, that proves (i). Substituting $h=g(k)$, one can write $\dfrac{p_{3}(h)}{p_{1}(h)}=\dfrac{\prod\limits_{j=1}^{3}(g(k)-x_{j})}{g(k)+\frac{a_{2}}{D}}.$ By \thetag{49}, $\prod\limits_{j=1}^{3}(Dx_{j}+a_{2})= \vert M \vert ^3 \prod\limits_{j>i}(m_{j}+m_{i})$. Using $f(x_{j})=v_{j}$, it is not hard to show that $g(k)-x_{j}=-\dfrac{(Dx_{j}+a_{2})(k-v_{j})}{Dk+a_{1}}\Rightarrow \prod\limits_{j=1}^{3}(g(k)-x_{j})=-\dfrac{\prod\limits_{j=1}^{3}(Dx_{j}+a_{2})}{(Dk+a_{1})^{3}}\prod\limits_{j=1}^{3}(k-v_{j})=-\vert M \vert ^3 \dfrac{\prod\limits_{j>i}(m_{j}+m_{i})}{(Dk+a_{1})^{3}}\prod\limits_{j=1}^{3}(k-v_{j})$. Finally, it follows easily that $g(k)+\dfrac{a_{2}}{D}=\vert M \vert ^2 \dfrac{\prod\limits_{j>i}(m_{j}+m_{i})}{-2D(Dk+a_{1})}$. Putting this altogether, and using $\dfrac{\vert M \vert}{D}=\dfrac{x_{2}-x_{3}}{m_{3}-m_{2}}$, one obtains $\dfrac{p_{3}(h)}{p_{1}(h)}=2\dfrac{x_{2}-x_{3}}{m_{3}-m_{2}}\dfrac{\prod\limits_{j=1}^{3}(k-v_{j})}{(k+\frac{a_{1}}{D})^{2}}=\dfrac{s_{3}(k)}{(k+\frac{a_{1}}{D})^{2}}$, where $s_{3}(k)=2\dfrac{x_{2}-x_{3}}{m_{3}-m_{2}}\prod\limits_{j=1}^{3}(k-v_{j})$. $\dfrac{p_{3}(h)}{p_{1}(h)}$ is positive, on $\gamma,$ precisely when $s_{3}(k)>0$ since $k+\dfrac{a_{1}}{D}=0$ cannot yield a point on $\gamma.$ Substituting $k=f(h),$ \thetag{50}, and arguing as above, one can also show that $\dfrac{q_{3}(k)}{q_{1}(k)}=-\dfrac{x_{2}-x_{3}}{m_{3}-m_{2}}\prod\limits_{j=1}^{3}m_{j}\dfrac{\prod\limits_{j=1}^{3}(h-w_{j})}{(h+\dfrac{a_{2}}{D})^{2}}$, which is positive, on $\gamma ,$ precisely when $r_{3}(h)$ $>0$ since $h+\dfrac{a_{2}}{D}=0$ cannot yield a point on $\gamma.$ Hence $$\Re =\{(h,k)\in \gamma: r_{3}(h)>0\text{ and }s_{3}(k)>0\}\tag{52}$$ Note that $\Re $ $\neq \emptyset $ since the incenter of $T$ is admissible. One could also prove directly that $\Re $ is nonempty. The \it uniqueness \rm follows from Theorem 1. Since $\dfrac{q_{3}(f(h))}{q_{1}(f(h))}=\dfrac{r_{3}(h)}{(p_{1}(h))^{2}}$, \thetag{38} follows from \thetag{52}. That proves (ii)(a). \nl 	
	We now prove that there is an $(h,k)\in \Re $ such that $E$ has eccentricity $e_{0},$with $m_{3}\neq 0.$ Using \thetag{35}, it follows easily that for any set of distinct $i,j,l\in \{1,2,3\}$, $r_{3}(x_{i})=\allowbreak \dfrac{1}{8}(m_{j}+m_{l})^{2}\dfrac{(m_{j}-m_{i})^{2}(m_{l}-m_{i})^{2}(x_{j}-x_{l})^{4}}{(m_{j}-m_{l})^{4}}$, which implies that $$r_{3}(x_{i})>0\text{ for any }i\tag{53}$$ and $r_{3}\left( -\dfrac{a_{2}}{D}\right) =\dfrac{1}{8}\left( m_{1}+m_{2}\right) ^{2}(x_{3}-x_{2})^{4}\left( m_{1}+m_{3}\right) ^{2}\dfrac{\left( m_{3}+m_{2}\right) ^{2}}{\left( m_{3}-m_{2}\right) ^{4}}$, which implies that $$r_{3}\left( -\dfrac{a_{2}}{D}\right) >0\tag{54}$$ Let $0<c\leq 1$ be given. $\dfrac{a^{2}}{b^{2}}=c\Rightarrow \dfrac{p_{1}(h)p_{3}(h)}{r_{3}(h)}=c,$ which holds if and only if $E(h)\equiv p_{1}(h)p_{3}(h)-cr_{3}(h)=0,$ $r_{3}(h)\neq 0.$ By \thetag{53} and \thetag{54}, $E(x_{j})=-cr_{3}(x_{j})<0$ and $E\left( -\dfrac{a_{2}}{D}\right) =-cr_{3}\left( -\dfrac{a_{2}}{D}\right) <0$. Let $x_{\max }=\max \{x_{1},x_{2},x_{3},-a_{2}/D\}$. Then $E(x_{\max })<0$, and since $\lim\limits_{h\rightarrow \infty }E(h)=\infty ,$ $E(h_{0})=0$ for some\ $h_{0}>x_{\max }.$ Since $p_{1}(h_{0})p_{3}(h_{0})>0$ it follows that $r_{3}(h_{0})\neq 0$ and $\dfrac{p_{1}(h_{0})p_{3}(h_{0})}{r_{3}(h_{0})}=c>0$ implies that $r_{3}(h_{0})>0$. Thus $(h_{0},f(h_{0}))\in \Re $ by 
\thetag{38}, which implies, upon letting $a^{2}=$ $\dfrac{p_{3}(h_{0})}{p_{1}(h_{0})}$and $b^{2}=$ $\dfrac{r_{3}(h_{0})}{(p_{1}(h_{0}))^{2}}$, that the equation $1-\dfrac{a^{2}}{b^{2}}=e_{0}^{2},0\leq e_{0}<1$ always has a solution. To prove (ii)(b), if $m_{3}=0$ then it still follows that $S_{12}=\dfrac{p_{3}(h)}{p_{1}(h)}.$ However$,\;y_{1}=y_{2}=c_{3}$, and $T_{12}=\dfrac{q_{3}(k)}{q_{1}(k)}$, where $q_{3}(k)=(k-c_{3})^{2}(k-y_{3}).$ Since $\dfrac{a_{1}}{D}=\allowbreak \dfrac{m_{1}c_{2}-m_{2}c_{1}}{m_{2}-m_{1}}=-y_{3},\;\dfrac{q_{3}(k)}{q_{1}(k)}=(k-c_{3})^{2}\Rightarrow T_{12}=\dfrac{q_{3}(f(h))}{q_{1}(f(h))}=(k-c_{3})^{2}>0$ when $k\neq c_{3}$. Now $k\neq c_{3}$ if and only if $h\neq g(c_{3})=\dfrac{m_{1}(c_{3}-c_{2})+m_{2}(c_{3}-c_{1})}{2m_{1}m_{2}}=\dfrac{1}{2}(x_{1}+x_{2}).$ That proves \thetag{39}. Again, the \it uniqueness \rm follows from Theorem 1. The proof that there is an $(h,k)\in \Re $ such that $E$ has eccentricity $e_{0}$ is similar to the proof above when $m_{3}\neq 0$ and we omit it. $\blacksquare$ \nl \nl
\bf \S 4.1 One of the Tangents is Vertical \rm 
\nl \nl \bf Theorem 13. \rm Let $L_{1}$ and $L_{2}\;$be distinct, non-concurrent, and non--vertical lines with equations $y=L_{j}(x)=m_{j}x+c_{j}$, $j=1,2,$ and assume that $0\neq m_{1}^{2}\neq m_{2}^{2}$. Let $L_{3}$ be the vertical line with equation $x=c_{3}$, and let $\gamma $ be the curve with equation $(h-x_{3})(k-a_{1})=-\dfrac{1}{2}\left( m_{1}+m_{2}\right) (c_{3}-x_{3})^{2},$ where $a_{1}=L_{2}(c_{3})+L_{1}(c_{3})-y_{3}.$ Let $q_{3}(k)=\prod\limits_{j=1}^{3}(k-y_{j}),q_{1}(k)=m_{1}^{2}(k-a_{1}).$ Then \nl
(i) $\Re =\left\{ (h,k)\in \gamma :h\neq c_{3}\text{ and }\dfrac{q_{3}(k)}{q_{1}(k)}>0\right\}$. For each $(h,k)\in \Re $, there is a unique ellipse $E:\dfrac{(x-h)^{2}}{a^{2}}+\dfrac{(y-k)^{2}}{b^{2}}=1$ which is tangent to each of the $L_{j}$, with $a^{2}=(h-c_{3})^{2}$and $b^{2}=$ $\dfrac{q_{3}(k)}{q_{1}(k)}.$ \nl 
(ii) Also, given any $0\leq e_{0}<1$, there is an $(h,k)\in \Re $ such that $E$ has eccentricity $e_{0}$. \nl 
\bf Proof. \rm Given $C=(h,k)$, write $L_{j}$ in the form $y-k=m_{j}(x-h)+b_{j}$, $j=1,2,$ and write $L_{3}$ in the form $x-h=b_{3}$. Arguing as earlier, a necessary condition for $E$ to be tangent to $L_{1}$ and $L_{2}$ is that \thetag{7} has the unique positive solution $(a^{2},b^{2})$ which satisfies \thetag{6}. Also, a necessary condition for $E$ to be tangent to the line $x=c_{3}$ is that $a\pm h=c_{3}\Rightarrow a^{2}=(c_{3}\pm h)^{2}=b_{3}^{2}>0\Rightarrow h\neq c_{3}.$ Thus $S_{12}=b_{3}^{2}\Rightarrow b_{2}^{2}-b_{1}^{2}-(m_{2}^{2}-m_{1}^{2})(h-c_{3})^{2}=0\Rightarrow -2(m_{2}-m_{1})hk-2(c_{2}-c_{1})k+2(m_{2}c_{2}-m_{1}c_{1}+c_{3}(m_{2}^{2}-m_{1}^{2}))h+c_{2}^{2}-c_{1}^{2}-c_{3}^{2}(m_{2}^{2}-m_{1}^{2})=0$ $\Rightarrow hk-\dfrac{c_{1}-c_{2}}{m_{2}-m_{1}}k-\dfrac{m_{2}c_{2}-m_{1}c_{1}+c_{3}(m_{2}^{2}-m_{1}^{2})}{m_{2}-m_{1}}h-\dfrac{c_{2}^{2}-c_{1}^{2}-c_{3}^{2}(m_{2}^{2}-m_{1}^{2})}{2(m_{2}-m_{1})}=0.$ Now 	
$\dfrac{m_{2}c_{2}-m_{1}c_{1}+c_{3}(m_{2}^{2}-m_{1}^{2})}{m_{2}-m_{1}}=-m_{1}x_{3}+c_{2}+c_{3}(m_{1}+m_{2})=L_{2}(c_{3})+L_{1}(c_{3})-L_{1}(x_{3})=a_{1}$, \ and thus we have $hk-x_{3}k-a_{1}h-a_{3}=0\Rightarrow (h-x_{3})(k-a_{1})=a_{1}x_{3}+a_{3}=$ 	
$\dfrac{m_{2}c_{2}-m_{1}c_{1}+c_{3}(m_{2}^{2}-m_{1}^{2})}{m_{2}-m_{1}}\dfrac{c_{1}-c_{2}}{m_{2}-m_{1}}+\dfrac{c_{2}^{2}-c_{1}^{2}-c_{3}^{2}(m_{2}^{2}-m_{1}^{2})}{2(m_{2}-m_{1})}=\allowbreak -\frac{1}{2}\left( m_{1}+m_{2}\right) \dfrac{\left( -c_{1}+c_{2}+(m_{2}-m_{1})c_{3}\right) ^{2}}{\left( m_{2}-m_{1}\right) ^{2}}=-\dfrac{1}{2}\left( m_{1}+m_{2}\right) (c_{3}-x_{3})^{2},$ which gives the curve $\gamma .$ $a^{2}=b_{3}^{2}\Rightarrow b^{2}=b_{1}^{2}-m_{1}^{2}a^{2}=b_{1}^{2}-m_{1}^{2}b_{3}^{2}>0\Rightarrow \dfrac{b_{1}^{2}}{m_{1}^{2}}>b_{3}^{2}\Rightarrow p(h)\equiv \dfrac{(\allowbreak m_{1}h+c_{1}-k)^{2}}{m_{1}^{2}}-(h-c_{3})^{2}>0.$ Solving $S_{12}-(h-c_{3})^{2}=\allowbreak 0$ for $h$ gives $h=\dfrac{1}{2}\dfrac{\allowbreak \left( 2c_{2}-2c_{1}\right) k-c_{2}^{2}+c_{1}^{2}+c_{3}^{2}(m_{2}^{2}-m_{1}^{2})}{\left( m_{1}-m_{2}\right) k+m_{2}c_{2}-m_{1}c_{1}+c_{3}(m_{2}^{2}-m_{1}^{2})}.$ Substituting into $p(h)$ yields $q(k)=0,$ where $q(k)=\left( c_{1}+m_{1}c_{3}-k\right) \left( c_{2}-k+m_{2}c_{3}\right) \dfrac{\allowbreak \left( m_{2}-m_{1}\right) k+m_{1}c_{2}-m_{2}c_{1}}{m_{1}^{2}(\left( m_{2}-m_{1}\right) k-(m_{2}c_{2}-m_{1}c_{1}+c_{3}(m_{2}^{2}-m_{1}^{2})))}.$ Thus $q(k)=\dfrac{q_{3}(k)}{q_{1}(k)},$ where $q_{3}(k)=\prod\limits_{j=1}^{3}(k-y_{j})$ and $q_{1}(k)=m_{1}^{2}(k-\dfrac{m_{2}c_{2}-m_{1}c_{1}+c_{3}(m_{2}^{2}-m_{1}^{2})}{m_{2}-m_{1}})=$ $m_{1}^{2}(k-(-m_{1}x_{3}+c_{2}+c_{3}(m_{1}+m_{2})))=m_{1}^{2}(k-a_{1}).$ That implies that $\Re $ $\subset \{(h,k)\in \gamma :h\neq c_{3}$ and $\dfrac{q_{3}(k)}{q_{1}(k)}>0\}.$ Now if $h\neq c_{3}$ and $\dfrac{q_{3}(k)}{q_{1}(k)}>0,$ then letting $a^{2}=(h-c_{3})^{2}$and $b^{2}=$ $\dfrac{q_{3}(k)}{q_{1}(k)}$ and basically reversing the steps above, $\dfrac{(x-h)^{2}}{a^{2}}+\dfrac{(y-k)^{2}}{b^{2}}=1$ is tangent to each of the $L_{j}$, $j=1,2,3.$ That proves (i). We omit the proof of (ii).  
$\blacksquare$ \nl \nl
\bf \S 4.2 Two of the tangents have slopes equal in absolute value \rm \nl
	In this section we assume that $m_{1}^{2}=m_{2}^{2}\neq m_{3}^{2}.$ Given $C=(h,k)$, write $L_{j}$ in the form $y-k=m_{j}(x-h)+b_{j}$, $j=1,2,3$. \thetag{42} now becomes $$\split m_{1}^{2}u+v=b_{1}^{2}\\ m_{1}^{2}u+v=b_{2}^{2}\\ m_{3}^{2}u+v=b_{3}^{2}\endsplit\tag{55}$$ Since rank $\!\left( \matrix m_1^2 & 1 \\ m_1^2 & 1 \\m_3^2 & 1 \endmatrix \right)=2,$ rank $\!\left( \matrix m_1^2 & 1 & b_1^2\\ m_1^2 & 1 & b_2^2\\m_3^2 & 1 & b_3^2\endmatrix \right) \geq 2,$ and $\Bigg\vert\matrix m_1^2 & 1 & b_1^2\\ m_1^2 & 1 & b_2^2\\m_3^2 & 1 & b_3^2\endmatrix\Bigg\vert=\allowbreak \left( m_{3}^{2}-m_{1}^{2}\right) \left( b_{2}^{2}-b_{1}^{2}\right) =0$ if and only if $b_{1}^{2}=b_{2}^{2},$ \thetag{55} has a solution if and only if $b_{1}^{2}=b_{2}^{2}$. The unique solution is $$u=S_{23}=S_{13}\tag{56}$$ and $$v=T_{23}=T_{13}\tag{57}$$ If \thetag{56} and \thetag{57} are positive, then we let $a^{2}=u$ and $b^{2}=v$ to obtain the equation of the required ellipse. The condition $b_{1}^{2}=b_{2}^{2}$ implies $(m_{1}h+c_{1}-k)^{2}=(m_{2}h+c_{2}-k)^{2}$, which implies(using $m_{1}^{2}=m_{2}^{2}$) that $$2(m_{2}-m_{1})hk+2(m_{1}c_{1}-\allowbreak m_{2}c_{2})h+2(c_{2}-c_{1})k+c_{1}^{2}-c_{2}^{2}=0\tag{58}$$ Our next theorem covers the case when $m_{2}=-m_{1}$. 
\nl \nl \bf Theorem 14. \rm Let $L_{1},L_{2},L_{3}$ be distinct, non-concurrent, and non--vertical lines with equations $y=L_{j}(x)=m_{j}x+c_{j}$, $j=1,2,3$. Assume that $m_{2}=-m_{1}\neq 0$ and that $m_{1}^{2}\neq m_{3}^{2}$. Let $Q(h)=(h-x_{1})(h-x_{2})$, $L_{1}(k)=-2\dfrac{x_{3}-x_{2}}{m_{3}+m_{1}}\left( k-\dfrac{m_{3}x_{3}+c_{3}+y_{3}}{2}\right) $, and $P(k)=(k-y_{1})(k-y_{2}).$
	
(a) $m_{3}\neq 0$: Let $L_{2}(h)=-2\dfrac{m_{1}m_{3}}{m_{3}+m_{1}}(y_{3}-y_{2})\left( h-\dfrac{m_{3}x_{3}+y_{3}-c_{3}}{2m_{3}}\right) $. Then $\Re =\Re _{1}\cup \Re _{2}$, where $\Re _{1}=\{(x_{3},k):L_{1}(k)>0$ and $P(k)>0\}$ and $\Re _{2}=\{(h,y_{3}):L_{2}(h)>0$ and $Q(h)>0\}.$ If $(h,k)\in \Re $, there is a unique ellipse $E:\dfrac{(x-h)^{2}}{a^{2}}+\dfrac{(y-k)^{2}}{b^{2}}=1$ which is tangent to each of the $L_{j}$, with 
$a^{2}=\left\{ \aligned L_{1}(k) \text { if } (h,k)\in \Re_{1}\\Q(h) \text { if } (h,k)\in \Re_{2} \endaligned $ and 
$b^{2}=\left\{ \aligned P(k) \, \text {if }(h,k)\in \Re_{1}\\L_2(h) \text {if }(h,k)\in \Re_{2} \endaligned $ \nl
(b) $m_{3}=0$: Then $\Re =\Re _{1}\cup \Re _{2},$where $\Re _{1}=\{(x_{3},k):L_{1}(k)>0$ and $k\neq c_{3}\},\Re _{2}=\{(h,y_{3}):h<\min (x_{1},x_{2})$ or $\max (x_{1},x_{2})<h\}.$ Again $E$ is unique, with $a^{2}$ as in (a), and $b^{2}=\left\{ \aligned (k-c_{3})^{2} \text {if }(h,k)\in \Re_{1}\\(y_{3}-c_{3})^{2} \text {if }(h,k)\in \Re_{2} \endaligned$ \nl
(c) Also, for (a) or (b) above, given any $0\leq e_{0}<1$, there is an $(h,k)\in \Re $ such that $E$ has eccentricity $e_{0}$. \nl
\bf Proof. \rm Since $m_{2}=-m_{1},\;x_{1}=\dfrac{c_{2}-c_{3}}{m_{1}+m_{3}}$, $x_{2}=\dfrac{c_{3}-c_{1}}{m_{1}-m_{3}}$, $\allowbreak $ $y_{1}=\allowbreak \dfrac{m_{1}c_{3}+m_{3}c_{2}}{m_{3}+m_{1}}$, $y_{2}=\dfrac{m_{1}c_{3}-m_{3}c_{1}}{m_{1}-m_{3}}$, $x_{3}=\dfrac{c_{2}-c_{1}}{2m_{1}}$, $y_{3}=\dfrac{c_{1}+c_{2}}{2},$ and \thetag{58} reduces to $-4m_{1}hk+2m_{1}(c_{1}+c_{2})h+2(c_{2}-c_{1})k+c_{1}^{2}-c_{2}^{2}=0\Rightarrow \allowbreak -\left( 2k-c_{1}-c_{2}\right) \left( 2m_{1}h-c_{2}+c_{1}\right) =0\Rightarrow (h-x_{3})(k-y_{3})=0\Rightarrow k=y_{3}$ or $h=x_{3}$. One can also obtain this equation by letting $m_{2}=-m_{1}$ in 
\thetag{36}. Hence the admissible centers must lie on $\gamma =\{(h,k):h=$ $x_{3}$ or $k=y_{3}\}$. Note that $\dfrac{a_{1}}{D}=\allowbreak -y_{3}$, $\dfrac{a_{2}}{D}=-x_{3}$, and $h=$ $x_{3}$ and $k=y_{3}$ are just the asymptotes of the hyperbola with equation \thetag{36}. Since $L_{3}(x_{2})=L_{1}(x_{2}),$ $\left( L_{3}(h)-L_{1}(h)\right) /(m_{3}-m_{1})=h-x_{2}.$ Write $L_{j}$ in the form $x=\bar{L}_{j}(y)=\dfrac{1}{m_{j}}y-\dfrac{c_{j}}{m_{j}}.$ Then $\bar{L}_{3}(y_{2})=\bar{L}_{1}(y_{2})\Rightarrow \left( \bar{L}_{3}(k)-\bar{L}_{1}(k)\right) /\left( 1/m_{3}-1/m_{1}\right) =k-y_{2}.$ By \thetag{51}, $S_{13}=(h-x_{2})\left( L_{3}(h)+L_{1}(h)-2k\right) /(m_{3}+m_{1})$. The equation of $\bar{L}_{j}$ can also be written as $x-h=\dfrac{1}{m_{j}}(y-k)-\dfrac{b_{j}}{m_{j}},m_{j}\neq 0$. Substituting $\dfrac{1}{m_{j}}$ for $m_{j}$ and $-\dfrac{b_{j}}{m_{j}}$ for $b_{j}$ in $S_{13}$ yields $T_{13}.$ Interchanging $h$ and $k$ then gives $T_{13}=(\bar{L}_{3}(k)-\bar{L}_{1}(k))(\bar{L}_{3}(k)+\bar{L}_{1}(k)-2h)/(1/m_{3}^{2}-1/m_{1}^{2})=(k-y_{2})(\bar{L}_{3}(k)+\bar{L}_{1}(k)-2h)/\left( 1/m_{3}+1/m_{1}\right) $ We must determine the points on $\gamma $ where $S_{13}$ and $T_{13}$ are \bf both positive \rm. We have two cases to consider. \nl
\bf Case 1: \rm $h=$ $x_{3}$ \nl
Then $S_{13}=(x_{3}-x_{2})(L_{3}(x_{3})+L_{1}(x_{3})-2k)/(m_{3}+m_{1})$ \nl $=-2(x_{3}-x_{2})(k-\frac{L_{3}(x_{3})+y_{3}}{2})=$ $$-2\dfrac{x_{3}-x_{2}}{m_{3}+m_{1}}\left( k-\frac{m_{3}x_{3}+c_{3}+y_{3}}{2}\right) =L_{1}(k)\tag{59}$$ Now $\bar{L}_{1}(y_{1})+x_{1}=2x_{3}$ and $\bar{L}_{3}(y_{1})=x_{1}\Rightarrow \bar{L}_{3}(k)+\bar{L}_{1}(k)-2x_{3}$ vanishes at $k=y_{1}\Rightarrow T_{13}=(k-y_{2})\left( \bar{L}_{3}(k)+\bar{L}_{1}(k)-2x_{3}\right) /\left( 1/m_{3}+1/m_{1}\right) =(k-y_{1})(k-y_{2})=P(k).$ Note that if $m_{3}=0,$ then $y_{1}=y_{2}=c_{3}$ and the above formula for $P(k)$ still holds since $T_{13}=(k-c_{3})^{2}.$ Also, $P(k)=\dfrac{q_{3}(k)}{q_{1}(k)}$ from Theorem 12 since $\dfrac{a_{1}}{D}=\allowbreak -y_{3}.$ \nl
\bf Case 2: \rm $k=y_{3}$ \nl
It also follows easily that $L_{3}(h)+L_{1}(h)-2y_{3}$ vanishes at $h=x_{1}.$ Then $S_{13}=(h-x_{2})\left( L_{3}(h)+L_{1}(h)-2y_{3}\right) /(m_{3}+m_{1})=(h-x_{1})(h-x_{2})=Q(h).$ Note that $Q(h)=\dfrac{p_{3}(h)}{p_{1}(h)}$ from Theorem 12 since $\dfrac{a_{2}}{D}=-x_{3}.$ Now $T_{13}=(y_{3}-y_{2})(\bar{L}_{3}(y_{3})+\bar{L}_{1}(y_{3})-2h)/\left( 1/m_{3}+1/m_{1}\right) =$ $(y_{3}-y_{2})(\bar{L}_{3}(y_{3})+x_{3}-2h)/\left( 1/m_{3}+1/m_{1}\right) =-2\dfrac{m_{1}m_{3}}{m_{3}+m_{1}}(y_{3}-y_{2})(h-\frac{\bar{L}_{3}(y_{3})+x_{3}}{2})=$ 
$$-2\dfrac{m_{1}m_{3}}{m_{3}+m_{1}}(y_{3}-y_{2})\left( h-\dfrac{m_{3}x_{3}+y_{3}-c_{3}}{2m_{3}}\right) =L_{2}(h),m_{3}\neq 0$$ If $m_{3}=0,$ and $h=x_{3},$ then $S_{13}=L_{1}(k)$ as above. $T_{13}=b_{3}^{2}=(L_{3}(x_{3})-k)^{2}>0$ if $k\neq L_{3}(x_{3})=c_{3}$. If $m_{3}=0,$ and $k=y_{3},$ then $S_{13}=Q(h)$ as above. $T_{13}=b_{3}^{2}=(L_{3}(h)-y_{3})^{2}=(c_{3}-y_{3})^{2}>0$ since if $c_{3}=y_{3},$ then $L_{1}(x_{3})=L_{2}(x_{3})=L_{3}(x_{3})=y_{3},$ which violates the assumption that the $L_{j}$ are not concurrent. The uniqueness in each case follows from Theorem 1 since $m_{1}^{2}\neq m_{3}^{2}$. That proves (a) and (b). To prove (c):
(i) If $h=x_{3}$, then $\dfrac{b^{2}}{a^{2}}=\dfrac{P(k)}{L_{1}(k)}$. Since $m_{2}=-m_{1},$ by \thetag{41} , 
$\vert M \vert =-m_{3}c_{2}+m_{1}c_{2}+m_{3}c_{1}-2m_{1}c_{3}+m_{1}c_{1}\neq 0.$ Some algebra shows that $L_{1}(y_{1})=\allowbreak \dfrac{\left( m_{1}c_{2}-m_{3}c_{2}+c_{1}m_{1}+m_{3}c_{1}-2m_{1}c_{3}\right) ^{2}}{4m_{1}^{2}\left( m_{3}+m_{1}\right) ^{2}}$ and $L_{1}(y_{2})=\dfrac{\left( m_{1}c_{2}-m_{3}c_{2}+c_{1}m_{1}+m_{3}c_{1}-2m_{1}c_{3}\right) ^{2}}{4m_{1}^{2}\left( m_{1}-m_{3}\right) ^{2}},$ which implies that $L_{1}(y_{j})>0,\;j=1,2.$ Let $0<c\leq 1$ be given. Setting $\dfrac{b^{2}}{a^{2}}=c$ implies that $\dfrac{P(k)}{L_{1}(k)}=c,$ which holds if and only if $E(k)\equiv P(k)-cL_{1}(k)=0,$ $L_{1}(k)\neq 0.$ Let $y_{\max }=\max \{y_{1},y_{2}\}$. Since $E(y_{j})=-cL_{1}(y_{j})<0,E(y_{\max })<0$, and since $\lim\limits_{k\rightarrow \infty }E(k)=\infty ,$ $E(k_{0})=0$ for some\ $k_{0}>y_{\max }.$ Since $P(k_{0})>0$ it follows that $L_{1}(k_{0})\neq 0$ and $\dfrac{P(k_{0})}{L_{1}(k_{0})}=c>0$ implies that $L_{1}(k_{0})>0$. Thus $(x_{3},k_{0})\in \Re $ by \thetag{38}, and implies, upon letting $b^{2}=P(k_{0})$ and $a^{2}=L_{1}(k_{0})$, that the equation $1-\dfrac{b^{2}}{a^{2}}=e_{0}^{2},0\leq e_{0}<1,$ always has a solution. \nl
(ii) If $k=y_{3}$, the proof follows in a similar fashion. $\blacksquare$ \nl 
Our next theorem covers the case when $m_{2}=m_{1}.$ 
\nl \nl \bf Theorem 15. \rm 
Let $L_{1},L_{2},L_{3}$ be distinct non--vertical lines with equations $y=L_{j}(x)=m_{j}x+c_{j}$, $j=1,2,3,$ and assume that $m_{2}=m_{1}$, with $m_{1}^{2}\neq m_{3}^{2}$. Let $L$ be the line with equation $y=L(x)=m_{1}x+\dfrac{1}{2}(c_{1}+c_{2})$. Let $P(h)=\left( \dfrac{m_{3}-m_{1}}{m_{3}+m_{1}}\right) (h-x_{1})(h-x_{2})$ and $Q(k)=-\left( \dfrac{m_{3}-m_{1}}{m_{3}+m_{1}}\right) (k-y_{1})(k-y_{2})$. \nl
(i) If $m_{1}\neq 0$, then $\Re =\left\{ (h,k):k=L(h)\text{ with }P(h)>0\text{ and }Q(k)>0\right\} $. \nl	
(ii) If $m_{1}=0$, then $\Re =\left\{ (h,\dfrac{1}{2}(c_{1}+c_{2})): P(h)>0 \right\} .$ \nl
In either case, if $(h,k)\in \Re ,$ then there is a unique ellipse $E:\dfrac{(x-h)^{2}}{a^{2}}+\dfrac{(y-k)^{2}}{b^{2}}=1$ which is tangent to each of the $L_{j}$, with $a^{2}=P(h)$ and $b^{2}=Q(k)$. \nl
(iii) Finally, given any $0\leq e_{0}<1$, there is an $(h,k)\in \Re $ such that $E$ has eccentricity $e_{0}$. 
\nl \bf Proof. \rm Note that $c_{1}\neq c_{2}$ since $L_{1}$ and $L_{2}$ are distinct. Since $m_{2}=m_{1}$,$\;x_{1}=\allowbreak \dfrac{c_{3}-c_{2}}{m_{1}-m_{3}}$, $x_{2}=\allowbreak \dfrac{c_{3}-c_{1}}{m_{1}-m_{3}}$, $\allowbreak $ $y_{1}=\allowbreak \dfrac{m_{1}c_{3}-m_{3}c_{2}}{m_{1}-m_{3}}$, $y_{2}=\allowbreak \dfrac{m_{1}c_{3}-m_{3}c_{1}}{m_{1}-m_{3}}$, and \thetag{58}  reduces to $\allowbreak 2hm_{1}(c_{1}-c_{2})+2k(c_{2}-c_{1})+c_{1}^{2}-c_{2}^{2}=0\Rightarrow $ $$2hm_{1}-2k+c_{1}+c_{2}=0\tag{60}$$ Note that by 
\thetag{60}, an admissible center must lie on $L$, and $L$ is parallel to, and lies exactly halfway between, $L_{1}$ and $L_{2}$. Solving \thetag{60} for $k$ yields $k=m_{1}h+\dfrac{1}{2}(c_{1}+c_{2})$. Since $L_{3}(x_{2})=L_{1}(x_{2}),$ and $L_{3}(x_{2})+L_{1}(x_{2})-2m_{1}x_{2}-(c_{1}+c_{2})=0,$ upon substituting for $k$ it follows that $S_{13}=(L_{3}(h)-L_{1}(h))(L_{3}(h)+L_{1}(h)-2m_{1}h-(c_{1}+c_{2}))$, which vanishes at $x_{1}$ and $x_{2}$. Thus $$S_{13}=\left( \dfrac{m_{3}-m_{1}}{m_{3}+m_{1}}\right) (h-x_{1})(h-x_{2})=P(h)\tag{61}$$ 
Solving \thetag{60} for $h$ yields(if $m_{1}\neq 0$) $$h=\dfrac{2k-(c_{1}+c_{2})}{2m_{1}}$$ It also follows easily that, upon substituting for $h\allowbreak $ $$T_{13}=-\left( \dfrac{m_{3}-m_{1}}{m_{3}+m_{1}}\right) (k-y_{1})(k-y_{2})=Q(k)\tag{62}$$ That proves (i). If $m_{1}=0$, then $(h,k)$ lies on the horizontal line with equation $y=\dfrac{1}{2}(c_{1}+c_{2})$. Since $y_{1}=c_{2}$ and $y_{2}=c_{1}$, $k=\dfrac{1}{2}(c_{1}+c_{2})\Rightarrow Q(k)=\dfrac{1}{4}(c_{1}-c_{2})^{2}>0$. Using \thetag{61}, the set of admissible centers now equals $\{(h,\dfrac{1}{2}(c_{1}+c_{2})):P(h)>0\},$ which proves (ii). The \it uniqueness \rm follows for each case from Theorem 1 since $m_{1}^{2}\neq m_{3}^{2}.$ To prove (iii): If $(h,k)\in \gamma $, then $\dfrac{b^{2}}{a^{2}}=\dfrac{Q(k)}{P(h)}=\dfrac{Q(L(h))}{P(h)}=\dfrac{Q(k)}{P(g(k))}$ and $m_{1}^{2}a^{2}+b^{2}=b_{1}^{2}\Rightarrow m_{1}^{2}P(h)+Q(k)=b_{1}^{2}=(m_{1}h+c_{1}-L(h))=\dfrac{1}{4}\left( c_{1}-c_{2}\right) ^{2}$. Letting $k=y_{j}$ yields $$P(g(y_{j}))=\dfrac{1}{4}\frac{\left( c_{1}-c_{2}\right) ^{2}}{m_{1}^{2}}>0\tag{63}$$ and letting $h=x_{j}$ yields $$Q(L(x_{j}))=\dfrac{1}{4}\left( c_{1}-c_{2}\right) ^{2}>0\tag{64}$$ 
\nl \it Case 1: \rm $\dfrac{m_{3}-m_{1}}{m_{3}+m_{1}}>0$ \nl
Let $0<c\leq 1$ be given. Setting $\dfrac{a^{2}}{b^{2}}=c$ implies that $\dfrac{P(h)}{Q(k)}=\dfrac{P(h)}{Q(L(h))}=c,$ which holds if and only if $E(h)\equiv P(h)-cQ(L(h))=0,$ $Q(L(h))\neq 0.$ Let $x_{\max }=\max \{x_{1},x_{2}\}$. Since $E(x_{j})=-cQ(L(x_{j}))<0$ by \thetag{64} ,$\;E(x_{\max })<0$. Since the coefficient of $h^{2}$ in $P(h)-cQ(L(h))$ is $\dfrac{m_{3}-m_{1}}{m_{3}+m_{1}}+c\dfrac{m_{3}-m_{1}}{m_{3}+m_{1}}m_{1}^{2}$, $\lim\limits_{h\rightarrow \infty }E(h)=\infty $. Thus $E(h_{0})=0$ for some\ $h_{0}>x_{\max }.$ Since $P(h_{0})>0$ it follows that $Q(L(h_{0}))\neq 0$ and $\dfrac{P(h_{0})}{Q(L(h_{0}))}=c>0$ implies that $Q(L(h_{0}))>0$. Letting $k_{0}=L(h_{0})$ implies that $(h_{0},k_{0})\in \Re $ by \thetag{38}, which implies, upon letting $a^{2}=P(h_{0})$ and $b^{2}=Q(L(h_{0}))$, that the equation $1-\dfrac{b^{2}}{a^{2}}=e_{0}^{2},0\leq e_{0}<1,$ always has a solution. 
\nl \it Case 2: \rm $\dfrac{m_{3}-m_{1}}{m_{3}+m_{1}}<0$ \nl 
Then consider $\dfrac{b^{2}}{a^{2}}=\dfrac{Q(k)}{P(h)}=\dfrac{Q(k)}{P(g(k))}$ and the rest of the proof follows in a fashion similar to case 1.  $\blacksquare$ \nl \nl
\bf \S 4.2 All three of the tangents have slopes equal in absolute value \rm \nl
	Clearly, if $m_{1}=m_{2}=m_{3}$, then there is no ellipse tangent to all of the $L_{j}$. Hence we can assume, without loss of generality, that $m_{1}=m_{2}=-m_{3}$. In this case the set of admissible centers consists of two points. For each such center, there are infinitely many ellipses tangent to the $L_{j}$. 
\nl \nl \bf Theorem 16. \rm Let $L_{j},\;j=1,2,3$ be three non--vertical lines with equations $y=m_{j}x+c_{j}$, and assume that $m_{1}=m_{2}=-m_{3}$. Then $\Re =\{(x_{2},y_{1}),(x_{1},y_{2})\}=\left\{ \left( \dfrac{c_{3}-c_{2}}{2m_{1}},\dfrac{c_{1}+c_{3}}{2}\right) ,\left( \dfrac{c_{3}-c_{1}}{2m_{1}},\dfrac{c_{2}+c_{3}}{2}\right) \right\} $. If $(h,k)\in \Re $, the ellipse $\dfrac{(x-h)^{2}}{a^{2}}+\dfrac{(y-k)^{2}}{b^{2}}=1$ is tangent to each of the $L_{j}$, where $a^{2}$ and $b^{2}$ are any positive solutions of $a^{2}m_{1}^{2}+b^{2}=\dfrac{1}{4}(c_{1}-c_{2})^{2}$. Finally, given any $0\leq e_{0}<1$ and any $(h,k)\in \Re $, one can choose $a^{2}$ and $b^{2}$ such that $E$ has eccentricity $e_{0}$. 
\nl \bf Proof. \rm Given $C=(h,k)$, write $L_{j}$ in the form $y-k=m_{j}(x-h)+b_{j}$, $j=1,2,3$. Since $m_{1}^{2}=m_{2}^{2}=m_{3}^{2},$ \thetag{42} has a solution if and only if $b_{1}^{2}=b_{2}^{2}=b_{3}^{2}.$ The solutions satisfy the one equation $$m_{1}^{2}u+v=b_{1}^{2}\tag{65}$$ If $b_{1}\neq 0$, then \thetag{65} has infinitely many positive solutions $(u,v)$ and the ellipse $E:\dfrac{(x-h)^{2}}{a^{2}}+\dfrac{(y-k)^{2}}{b^{2}}=1$ is tangent to $L_{1},L_{2},$ and $L_{3}$, where $a^{2}=u$ and $b^{2}=v$. By \thetag{8}, $b_{1}^{2}=b_{2}^{2}$ and $b_{1}^{2}=b_{3}^{2}$ imply that $2(m_{j}-m_{1})hk+2(m_{1}c_{1}-\allowbreak m_{j}c_{j})h+2(c_{j}-c_{1})k+c_{1}^{2}-c_{j}^{2}=0$ for $j=2,3$, which yields the two simultaneous equations $$2hm_{1}-2k+c_{1}+c_{2}=0,\;\left( 2k-c_{1}-c_{3}\right) \left( 2m_{1}h-c_{3}+c_{1}\right) =0$$ with solutions $(h,k)=\left( \dfrac{c_{3}-c_{2}}{2m_{1}},\dfrac{c_{1}+c_{3}}{2}\right) =(x_{1},y_{2})$ and $\left( \dfrac{c_{3}-c_{1}}{2m_{1}},\dfrac{c_{2}+c_{3}}{2}\right) =(x_{2},y_{1})$. Note that those points are distinct since $c_{1}\neq c_{2}$(else $L_{1}=L_{2}$). It is not hard to show that $(h,k)=\left( \dfrac{c_{3}-c_{2}}{2m_{1}},\dfrac{c_{1}+c_{3}}{2}\right) $ if and only if $b_{1}=\dfrac{1}{2}(c_{1}-c_{2})$,\ $b_{2}=-b_{1}$, and $b_{3}=-b_{1}$, and $(h,k)=\left( \dfrac{c_{3}-c_{1}}{2m_{1}},\dfrac{c_{2}+c_{3}}{2}\right) $ if and only if $b_{1}=\dfrac{1}{2}(c_{1}-c_{2})$,\ $b_{2}=-b_{1}$, and $b_{3}=b_{1}.$ For each choice of $(h,k)$ above, $b_{1}\neq 0$ with $b_{1}^{2}=b_{2}^{2}=b_{3}^{2}$. As $a^{2}$ varies between $0$ and $\dfrac{b_{1}^{2}}{m_{1}^{2}}$, one obtains infinitely many positive solutions $a^{2},b^{2}$ of the equations in \thetag{65}, and $\dfrac{a^{2}}{b^{2}}$ varies from $0$ to $\infty.$ Thus one can also find an ellipse with any preassigned eccentricity. $\blacksquare$ \nl 
\nl \bf Remark. \rm For three given tangents(no two of which are parallel) there is a very nice result about ellipses inscribed in the triangle, $T,$ formed by the tangents. Let $T_{M}$ denote the midpoint triangle, that is the triangle whose vertices are the midpoints of the sides of $T$. Chakerian \cite{1} proved that $\cup _{-\frac{\pi }{2}<\alpha <\frac{\pi }{2}}\Re _{\alpha }=int(T_{M})$. Chakerian uses orthogonal projection to prove this result, methods much different than the ones used in this paper. One can also prove this result using methods similar to the proofs given for four given lines, though the proof is longer and not as concise. However, using our methods one can also show that the union, over $\alpha ,$ of the $\alpha $ admissible centers for \bf hyperbolas \rm equals $int(T)-\{int(T_{M})\cup T_{M}\}$. \nl \nl 
\Refs
\ref\no 1\by Chakerian, G.D.:\book A Distorted View of Geometry, Mathematical Plums
\publ MAA\publaddr Washington, D.C.\yr1979\pages 130--150\endref
\ref\no 2\by Dorrie, Heinrich.:\book 100 Great Problems of Elementary Mathematics, 
\publ Dover\publaddr New York.\yr1965\endref
\ref\no 3\by Horwitz, Alan:\pages 122--151\paper Finding an Ellipse Tangent to Finitely Many Given Lines
\yr2000\vol 2\jour Southwest Journal of Pure and Applied Mathematics\endref


\endRefs

\enddocument




	








	
