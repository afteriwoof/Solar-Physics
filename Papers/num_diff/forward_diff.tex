\section{Forward differencing}
\label{sect:forward}

The forward differencing technique involves the computation of the derivative at the point $t + \Delta t$ by extrapolating forward from the point $t$. This uses the Taylor series:
\begin{equation}
r(t + \Delta t) = r(t) + r'(t)\Delta t +  \frac{r''(t)}{2!}(\Delta t)^{2} + \frac{r'''(t)}{3!}(\Delta t)^{3}  + ...
\end{equation}
This equation can be re-arranged to give
\begin{equation}
r'(t)\Delta t = r(t + \Delta t) - r(t) -  \frac{r''(t)}{2!}(\Delta t)^{2} - \frac{r'''(t)}{3!}(\Delta t)^{3}  + ...
\end{equation}
which then gives
\begin{equation}
r'(t) = \frac{r(t + \Delta t) - r(t)}{\Delta t} -  \frac{r''(t)}{2!}(\Delta t) - \frac{r'''(t)}{3!}(\Delta t)^{2}  + ...
\end{equation}
This is usually written as
\begin{equation}
r'(t) = v = \frac{r(t + \Delta t) - r(t)}{\Delta t} + O(\Delta t)
\end{equation}
where $O(\Delta t)$ is the truncation error term, determined by the distance between neighbouring points ($\Delta t$). This technique assumes a straight line gradient between points. 

This estimate of the velocity is dependent on the initial units used for the distance. In the case of the simulation work done here, the units of distance are mega-metres (1~Mm = $10^6$~m). This produces an estimate of velocity in units of Mm~s$^{-1}$. To convert this to acceptable units of km~s$^{-1}$ requires multiplying the estimated velocity values by $10^{3}$. This has been done in all plots showing the numerically derived velocity. Similarly, converting the acceleration from units of Mm~s$^{-2}$ requires multiplying the estimated acceleration values by $10^{6}$.

%The forward-difference technique may be used to obtain the velocity and acceleration of the data numerically, without the use of fits. If the data can be modeled as a linear function of the form $r(t) = r_0 + v_0 t$, with a noise term of the form $\delta r$ added to the distance data (i.e.:\ $r(t) = r_0 + v_0 t + \delta r$), the velocity of the data can be estimated as
%\begin{equation}
%v = v_{0} + \frac{\delta r_{(t + \Delta t)} - \delta r_{t}}{\Delta t}
%\end{equation} 
%In this case, the velocity is estimated as the initial velocity $v_0$ with a correction term that accounts for the variation in the original data-set due to noise. This correction term is unique to each data-point due to the random nature of the applied noise.

It is possible to derive the value of the truncation error term in terms of the original $r(t)$ values. The truncation error is given as
\begin{equation}
O(\Delta t) = \frac{r''(t)}{2!}(\Delta t)
\end{equation}
The $r''(t)$ term may be decomposed using the original forward-difference definition:
\begin{equation}
r''(t) = \frac{r'(t + \Delta t) - r'(t)}{\Delta t}
\end{equation}
Rewriting each term using the original functional forms produces
\begin{equation}
O(\Delta t) = \frac{r(t + 2\Delta t) - 2r(t + \Delta t) + r(t)}{2!\Delta t}
\end{equation}
The error term associated with the velocity estimate using the forward-difference technique is therefore dependent on the value of the function $r(t)$ at the points $t$, $t+\Delta t$ and $t+2\Delta t$.

This equation must be modified when dealing with the error associated with the acceleration estimate. In this case, the truncation error term would be given as:
\begin{equation}
O(\Delta t) = \frac{v(t + 2\Delta t) - 2v(t + \Delta t) + v(t)}{2!\Delta t}
\end{equation}
Here, the velocity function $v(t)$ is treated as the base function, rather than the distance function $r(t)$ as above. 

The forward-difference technique is a very simplistic technique that produces spiky plots, with large variation between points. This is a result of the inherent assumption made by the forward-difference technique that there is a straight-line gradient between points. In addition, the forward-difference technique removes a point from the end of the data-set with each differentiation due to the way it calculates the derivative.